\documentclass[../../header.tex]{subfiles}

\begin{document}

\problem{\textit{Magnetic field due to three solenoid coils.} 
Three identical solenoidal coils, wound uniformly and densely with $N$ turns of thin wire, are positioned in space as shown in Fig.~Q4.4. The axes of coils lie in the same plane and the permeability everywhere is $\mu_o$. Let $I_1$, $I_2$ and $I_3$ denote the intensities of time-invariant currents in the coils. Consider the following two cases: (a) $I_1 = I_2 = I_3 = I$ and (b) $I_1 = I, I_2 = I_3 = 0$. If $I > 0$, the magnetic flux density at the center of the system (the point $P$) for case (a) is
\begin{center}
\includegraphics[width=0.8\textwidth]{\npath Q4-6.png}
\end{center}
\begin{enumerate}[label=(\Alph*)]\setlength{\parskip}{1pt}
\item larger than
\item the same as
\item {smaller than}
\end{enumerate}
the magnetic flux density at the same point for case (b).}

\solution{(C)}
\answer{(C)}

\end{document}




































