\documentclass[../../header.tex]{subfiles}

\begin{document}

\problem{\textit{Computing the motional emf in a loop.} A planar metallic wire loop moves with a velocity $\vect{v}$ in a nonuniform static magnetic field of flux density $\vect{B}$, as depicted in Fig.~Q6.8. The magnetic field due to the induced current in the loop is negligible. Consider the following two expressions computed for this loop: $A_1 = -\dfrac{d}{dt} \int_{S} \vect{B} \cdot d\vect{S}$ and $A_2 = \oint_C (\vect{v} \times \vect{B})\cdot d\vect{l}$, where the reference directions of $d\vect{l}$ and $d\vect{S}$ are interconnected by the right-hand rule. Which of the following is true for the induced emf, $e_\mathrm{ind}$, in the loop?
\begin{enumerate}[label=(\Alph*)]\setlength{\parskip}{1pt}
\item $e_\mathrm{ind} = A_1 + A_2$.
\item $e_\mathrm{ind} = A_1 - A_2$.
\item {$e_\mathrm{ind} = A_1 = A_2$.}
\item $e_\mathrm{ind} = A_1$ and $e_\mathrm{ind} \ne A_2$.
\item $e_\mathrm{ind} = A_2$ and $e_\mathrm{ind} \ne A_1$.
\end{enumerate}
\begin{center}
\includegraphics[scale=0.7]{\cpath Q6-8.png}
\end{center}}

\solution{(C)}
\answer{(C)}

\end{document}




































