\documentclass[../../header.tex]{subfiles}

\begin{document}
%\itemi{Question 1.23 (Notaros) 
\problem{\textit{Negative point charge in a Faraday case.} A negatively charged small body is situated inside an uncharged spherical metallic shell. The distribution of induced charges on the outer surface of the shell can be represented as in 
\begin{enumerate}[label=(\Alph*)]\setlength{\parskip}{1pt}
\item Fig.~Q1.17(a)
\item {Fig.~Q1.17(b)}
\item Fig.~Q1.17(c)
\item Fig.~Q1.17(d)
\item Fig.~Q1.17(e)
\end{enumerate}
\begin{center}
	\includegraphics[scale = 0.7]{\npath Q1-17.jpg}
\end{center}}

\solution{(B)}
\answer{(B)}
\end{document}




































