\documentclass[../../header.tex]{subfiles}

\begin{document}

%\itemi{Question 1.12 (Notaros) 
\problem{\textit{Flux of the electric field vector through an infinite
		surface.}
A point charge Q is situated in free space at a very small hight h (h $\rightarrow$ 0)
above an imaginary (nonmaterial) infinite flat surface S, as depicted in Fig.Q1.9.
The surface is oriented upward. The flux of the electric field intensity vector due to
the charge Q through S comes out to be
\begin{enumerate}[label=(\Alph*)]\setlength{\parskip}{1pt}
\item $\Psi_E = Q/(4\pi\epsilon_0)$.
\item $\Psi_E = Q/(2\epsilon_0)$.
\item $\Psi_E = -Q/(2\epsilon_0)$.
\item $\Psi_E = -Q/\epsilon_0$.
\item $\Psi_E = 0$.
\item $\Psi_E \rightarrow \infty$.
\end{enumerate}
\includegraphics[scale=0.5]{\npath 1-12_diagram.png}}

\solution{(C)}
\answer{(C)}
\end{document}




































