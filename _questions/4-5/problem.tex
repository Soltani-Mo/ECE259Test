\documentclass[../../header.tex]{subfiles}

\begin{document}

\problem{\textit{Magnetic field of a rectangular current loop}. A rectangular wire loop of edge lengths $a$ and $b$ in air carries a steady current of intensity $I$ ($I > 0$), as shown in Fig.~Q4.3. The magnetic flux density vector $\vec{B}$ at the point $M$ in the figure can be represented as
\begin{center}
\includegraphics[width=0.9\textwidth]{\npath Q4-5.png}
\end{center}
\begin{enumerate}[label=(\Alph*)]\setlength{\parskip}{1pt}
\item $\vec{B} = B_x \vec{a}_x$, where $B_x > 0$.
\item $\vec{B} = B_x \vec{a}_x$, where $B_x < 0$.
\item {$\vec{B} = B_z \vec{a}_z$, where $B_z > 0$.}
\item $\vec{B} = B_z \vec{a}_z$, where $B_z < 0$.
\item $\vec{B} = B_x \vec{a}_x + B_y \vec{a}_y$, where $B_x \neq 0$ and $B_y \neq 0$.
\end{enumerate}}

\solution{(C)}
\answer{(C)}

\end{document}




































