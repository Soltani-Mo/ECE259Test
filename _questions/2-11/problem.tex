\documentclass[../../header.tex]{subfiles}

\begin{document}

\problem{\textit{Capacitance of a ``cubical'' capacitor.} Consider a ``cubical'' capacitor, which consist of two concentric hollow metallic cubes with thin walls, as shown in Fig.~Q2.5. The edge lengths of the inner and outer conductors are $a=5$\,cm and $b=15$\,cm, respectively, and the medium between the conductors is air. If $a$ is made twice larger and $b$ kept the same, the capacitance of the capacitor
\begin{enumerate}[label=(\Alph*)]\setlength{\parskip}{1pt}
\item {increases.}
\item decreases.
\item remains the same.
\end{enumerate}
\begin{center}
\includegraphics[scale=0.35]{\npath Q2-11.png}
\end{center}}

\solution{(A)}
\answer{(A)}

\end{document}




































