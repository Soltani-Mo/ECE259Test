\documentclass[../../header.tex]{subfiles}

\begin{document}

\problem{\textit{Coaxial cable partly filled with ferromagnetic layer.} A coaxial cable carries a time-invariant current $I$. A thin layer of a ferromagnetic material is placed near the outer conductor, and the rest of the space between the conductor is air-filled. With respect to the notation in the Fig.~Q5.1 showing the cable cross section, the magnetic field exist only in
\begin{center}
\includegraphics[width=0.7\textwidth]{\npath Q5-7_notaros.png}
\end{center}
\begin{enumerate}[label=(\Alph*)]\setlength{\parskip}{1pt}
\item region 3.
\item regions 2 and 3.
\item regions 1, 3 and 4.
\item {regions 1, 2, 3 and 4.}
\item regions 1, 2, 3, 4 and 5.
\end{enumerate}}

\solution{(D)}
\answer{(D)}

\end{document}




































