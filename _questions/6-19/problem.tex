\documentclass[../../header.tex]{subfiles}

\begin{document}

\problem{\textit{Moving loop in a magnetic field.} A rectangular loop moves with a constant velocity $v$ ($\vect{v} = v \uvect{a}_x$) in a magnetic field of flux density vector $\vect{B}$. The ambient medium is air. Referring to Fig.~Q6.7, and with $B_0$, $\omega$, and $a$ being positive constants, there is a nonzero emf induced in the loop ($e_\mathrm{ind} \ne 0$) if 
\begin{enumerate}[label=(\Alph*)]\setlength{\parskip}{1pt}
\item $\vect{B} = B_0 \cos \omega t \, \uvect{a}_x$.
\item $\vect{B} = B_0 \cos \omega t \, \uvect{a}_z$.
\item $\vect{B} = B_0 \, \uvect{a}_z$.
\item $\vect{B} = B_0 x \, \uvect{a}_z / a$.
\item $\vect{B} = B_0 y \, \uvect{a}_z / a$.
\item {More than one of the above cases}.
\end{enumerate}
\begin{center}
\includegraphics[scale=0.6]{\cpath Q6-7.png}
\end{center}}

\solution{(F)}
\answer{(F)}

\end{document}




































