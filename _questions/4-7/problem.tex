\documentclass[../../header.tex]{subfiles}

\begin{document}

\problem{\textit{Amperian contour outside a current conductor.} A time-invariant current of intensity $I$ ($I > 0$) is established in a cylindrical copper conductor. The conductor is situated in air. The circulation (line integral) of the magnetic flux density vector, $\vec{B}$, along a contour $C$ composed from two circular and two radial parts and positioned outside the conductor, as shown in Fig.~Q4.5, is
\begin{center}
\includegraphics[width=0.8\textwidth]{\npath Q4-7.png}
\end{center}
\begin{enumerate}[label=(\Alph*)]\setlength{\parskip}{1pt}
\item $\mu_0 I$
\item $-\mu_0 I$
\item greater than $\mu_0 I$
\item positive and less than $\mu_0 I$
\item {zero}
\end{enumerate}
($\mu_0$ is the permeability of vacuum).}

\solution{(E)}
\answer{(E)}

\end{document}




































