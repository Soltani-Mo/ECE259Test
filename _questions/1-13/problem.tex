\documentclass[../../header.tex]{subfiles}

\begin{document}

%\itemi{Question 1.12 (Notaros) 
\problem{\textit{Flux through a cube side, charge at a vertex.}
A point
charge $Q$ is located at one of the vertices of an imaginary cube in free space, as
shown in Fig.Q1.10. The outward flux of the electric field intensity vector due to
this charge through a cube side that does not contain the charge (e.g., the upper
cube side in the figure) equals
\begin{enumerate}[label=(\Alph*)]\setlength{\parskip}{1pt}
\item $\Psi_E = Q/\epsilon_0$.
\item $\Psi_E = Q$.
\item $\Psi_E = Q/(2\epsilon_0)$.
\item $\Psi_E = Q/(6\epsilon_0)$.
\item $\Psi_E = Q/(24\epsilon_0)$.
\item $\Psi_E = 0$.
\end{enumerate}
\includegraphics[scale=0.5]{\npath 1-13_diagram.png}}

\solution{(E)}
\answer{(E)}
\end{document}




































