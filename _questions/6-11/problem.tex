\documentclass[../../header.tex]{subfiles}

\begin{document}

\problem{\textit{Wire loop in a uniform time-harmonic magnetic field.} A rectangular wire loop is situated in a uniform low-frequency time-harmonic magnetic field of flux density $B(t) = B_0 \sin \omega t$ ($B_0 > 0$). The vector $\vect{B}$ is perpendicular to the plane of the loop, as shown in Fig.Q6.3. The magnetic field due to induced currents can be neglected. The induced emf in the loop is of the following form (${\cal E}_0$ is positive constant):
\begin{center}
%\includegraphics[scale = 0.8]{\cpath Q6-3.png}
\includegraphics[scale = 0.95]{\cpath Q6-11.pdf}
\end{center}
\begin{enumerate}[label=(\Alph*)]\setlength{\parskip}{1pt}
\item $e_\mathrm{ind}(t) = {\cal E}_0 \sin \omega t$.
\item $e_\mathrm{ind} (t) = -{\cal E}_0 \cos 2\omega t$.
\item $e_\mathrm{ind} (t) = - {\cal E}_0$.
\item {$e_\mathrm{ind} (t) = - {\cal E}_0 \cos \omega t$}.
\item $e_\mathrm{ind}(t) = 0$.
\end{enumerate}}

\solution{(D)}
\answer{(D)}

\end{document}




































