\documentclass[../../header.tex]{subfiles}

\begin{document}

\problem{\textit{Approximations in analysis of magnetic circuits.} In analysis of magnetic circuits, a set of approximations is introduced to simplify the computation. However, consider the following possible assumptions: (a) The magnetic flux is concentrated exclusively inside the branches of the ferromagnetic core and air gaps; (b) Magnetic materials of the core can be considered to be linear; (c) Magnetic materials of the branches are never in the state of saturation; (d) There is no magnetic field in air gaps; (e) Lengths of air gaps can be considered to be zero; (f) The fringing magnetic flux near the edges of air gaps can be neglected; (g) The magnetic field intensity ($H$) is the same in all branches of the circuit; (h) The magnetic field is uniform throughout the volume of each branch of the circuit; (i) Magnetic fluxes are the same in all branches of the circuit. Which of the assumptions constitute the set of approximations used in the analysis?
\begin{enumerate}[label=(\Alph*)]\setlength{\parskip}{1pt}
\item All assumptions, (a) - (i).
\item Assumptions (a)-(c) and (g)-(i).
\item Assumptions (b)-(e) and (i).
\item Assumptions (a), (e), and (g).
\item {Assumptions (a), (f), and (h).}
\end{enumerate}}

\solution{(E)}
\answer{(E)}

\end{document}




































