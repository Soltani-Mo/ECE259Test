\documentclass[../../header.tex]{subfiles}

\begin{document}

%\itemi{Question 1.8 (Notaros) 
\problem{\textit{Contour in the field of a point charge.}
A point charge $Q$ is situated in free space. The line integral (circulation) of the electric field intensity vector $\vec{E}$ due to this charge alone the contour $C$ in Fig.~Q1.7, composed of two circular parts of radii $a$ and $2a$, respectively, and two radial parts of length $a$, amounts to ($\varepsilon_o$ is the permittivity of a vacuum)
\begin{enumerate}[label=(\Alph*)]\setlength{\parskip}{1pt}
\item $Q/(4\pi \varepsilon_o a)$.
\item $-Q/(4\pi \varepsilon_o a)$.
\item $Q/(8 \varepsilon_o a)$.
\item $-Q/(8 \varepsilon_o a)$.
\item zero.
\end{enumerate}
\includegraphics[scale=0.8]{\npath 1-8_diagram.png}}

\solution{(E)}
\answer{(E)}
\end{document}




































