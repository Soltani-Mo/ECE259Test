\documentclass[../../header.tex]{subfiles}

\begin{document}

\problem{\textit{Induced current in a superconducting contour.} Repeat Question 6.12 (given below) but for a superconducting  contour brough in a uniform magnetostatic field.\\
\textit{Magnetic flux due to the induced current in a contour}. A rectangular copper contour of area $S$ is first situated outside any magnetic field, and there is no current in it. The contour is then brought in a uniform time-invariant magnetic field of flux density $B$ and positioned so that the vector $\vect{B}$ is perpendicular to the plane of the contour (like the situtation in Fig.~Q6.3). In the new steady state, the magnetic flux through the contour due to the current induced in it, computed with respect to the same orientation as that of the vector $\vect{B}$, equals
\begin{enumerate}[label=(\Alph*)]\setlength{\parskip}{1pt}
\item $\Phi_\mathrm{ind} = BS$\,,
\item {$\Phi_\mathrm{ind} = -BS$}\,,
\item $\Phi_\mathrm{ind} = kBS$\,,
\item $\Phi_\mathrm{ind} = -kBS$\,,
\item $\Phi_\mathrm{ind} = 0$\,,
\end{enumerate}
where $k$ is a dimensionless constant and $0<k<1$.
\begin{center}
\includegraphics[scale=0.9]{\cpath Q6-3.pdf}
\end{center}}

\solution{(B)}
\answer{(B)}

\end{document}




































