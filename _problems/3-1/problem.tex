\documentclass[../../header.tex]{subfiles}

\begin{document}

\problem{A point charge $2Q$ is placed at the center of an air-filled spherical metallic (perfectly conducting) shell, charged with $Q$ and situated in air. The inner and outer radii of the shell are $a$ and $b$ ($a<b$). What is the total charge on the inner and outer surface of the shell, respectively? Find the potential of the shell.}

\solution{
Every field line from the inner point charge is terminated at a negatively charged point on the metallic shell. We also know that there are no volumetric charges within conductors at electrostatic equilibrium. As a result, the $Q$ of charge on the outer conductor will redistribute itself when the point charge is placed within. This will mean that a surface charge of $-2Q$ is induced on the inner part of the outer sphere, while an outer charge of $3Q$ is induced to maintain the $Q$ of charge originally on the outer shell.

The electric field intensity outside the sphere, using Gauss' law, is
\begin{align*}
\vect{E} = \frac{3Q}{4\pi\varepsilon_0 R^2}\uvect{a}_R \;\; b\leq R\leq\infty.
\end{align*}
Therefore we can solve for the potential as
\begin{align*}
V &= -\int_\infty^b \frac{3Q}{4\pi\varepsilon_0 R^2} dr\\
V &= \frac{3Q}{4\pi \varepsilon_0 b}.
\end{align*}
}

\answer{
Inner charge is $-2Q$, outer charge is $3Q$, potential of the shell is $3Q/(4\pi \varepsilon_0 b)$.}
\end{document}




































