\documentclass[../../header.tex]{subfiles}

\begin{document}

\problem{[Cheng P.3-27] What are the boundary conditions that must be satisfied by the electric potential at an interface between two perfect dielectrics with dielectric constants $\varepsilon_{r1}$ and $\varepsilon_{r2}$?}

\solution{
First, we know that the electrostatic potential $V$ is continuous across a boundary. To see this, let's find the voltage difference between $a$ and $b$, two points on either side of the boundary a length $\Delta$ apart, in the presence of a constant $\vect{E}$.
\begin{align*}
V_{ab} &= -\int_a^b \vect{E}\cdot dl\\
V_{ab} &= -{E}_0(b-a)\\
V_{ab} &= -{E}_0(\Delta).
\end{align*}
Now if we let $\Delta \rightarrow 0$, then $V_{ab}\rightarrow 0$. So if we have a voltage $V_1$ on one side of the boundary and $V_2$ on the other we can say
\begin{align*}
V_1 &= V_2,
\end{align*}
Futhermore, we know from our electric field boundary conditions we can say 
\begin{align*}
\varepsilon_1 \vect{E}_{n1}&=\varepsilon_2\vect{E}_{n2}\\  \vect{E} &= -\nabla V\\
\varepsilon_1 \frac{\partial V_1}{ \partial n} &= \varepsilon_2 \frac{\partial V_2}{ \partial n}.  
\end{align*}}
\answer{
\begin{align*}
\varepsilon_1 \frac{\partial V_1}{ \partial n} &= \varepsilon_2 \frac{\partial V_2}{ \partial n}  \\[0.25em]
V_1 &= V_2
\end{align*}}

\end{document}




































