\documentclass[../../header.tex]{subfiles}

\begin{document}
% Same as Workbook 6-4
\problem{[Cheng P.6-11] A long wire carrying a current $I$ folds back with a semicircular bend of radius $b$ as in Fig. 6-38. Determine the magnetic flux density at the center point $P$ of the bend.
\begin{center}
\includegraphics[scale=0.6]{\ppath cheng6-11_prompt.png}
\end{center}
}

\solution{This problem is the superposition of two problems:
\begin{align*}
\vect{B} = \vect{B}_1+\vect{B}_2,
\end{align*}
where 
\begin{enumerate}
\item $\vect{B}_1$ is the magnetic flux density at $P$ due to two semi-infinite wires carrying equal and opposite currents. 

Assuming $\uvect{a}_z$ points out of the page:
\begin{align*}
\vect{B}_1 = \uvect{a}_z\frac{\mu_0I}{2\pi b}.
\end{align*}
\item $\vect{B}_2$ is the magnetic flux density at $P$ due to a half circle. Taking one half of the result in Eq. (6-38) for $z=0$
\begin{align*}
\vect{B}_2 = \uvect{a}_z \frac{\mu_0I}{4b}
\end{align*}
\end{enumerate}
Therefore,
\begin{align*}
\vect{B} = \uvect{a}_z\frac{\mu_0I}{2b}\left( \frac{1}{\pi} +\frac{1}{2}\right).
\end{align*}
}

\answer{\begin{align*}
\vect{B} = \uvect{a}_z\frac{\mu_0I}{2b}\left( \frac{1}{\pi} +\frac{1}{2}\right)
\end{align*}}
\end{document}




































