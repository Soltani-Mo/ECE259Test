\documentclass[../../header.tex]{subfiles}

\begin{document}

\problem{[Cheng P.6-37] Calculate the mutual inductance per unit length between two parallel wire transmission lines $A-A'$ and $B-B'$ separated by a distance $D$, as shown in Fig. 6-47. Assume the wire radius is much smaller than $D$ and the wire spacing $d$.
\begin{center}
\includegraphics[scale=1]{\ppath Cheng6-37_prompt.png}
\end{center}}

\solution{ The magnetic flux density is a function of distance from the wire as shown in the diagram below.
\begin{center}
\includegraphics[scale=0.4]{\ppath Cheng6-37_diagram.png}
\end{center}
$\vect{B}$ at a distance $r$ from an infinitely long line carrying current $I$ is $\vect{B} = \uvect{a}_\phi\frac{\mu I}{2\pi r}$.

For a unit length flux due to $I$ in line $A$ that links with the second line pair $B-B'$ is
\begin{align*}
\Phi_A' = \frac{\mu_0 I}{2\pi} \int_{AB}^{AC} \frac{\text{d}r}{r} = \frac{\mu_0 I}{2\pi}\ln\frac{AC}{AB}.
\end{align*}
That unit length flux due to $A'$ is 
\begin{align*}
\Phi_{A'}' = \frac{\mu_0 I}{2\pi}\ln\frac{A'C'}{A'B'}.
\end{align*}
The total flux linkage is therefore
\begin{align*}
\Lambda_{12}' &=  \Phi_{A'}+\Phi_{A'}'\\
&=  \frac{\mu_0 I}{2\pi}\ln\frac{(AC)(A'C')}{(AB)(A'B')}\\
&=  \frac{\mu_0 I}{2\pi}\ln\frac{(AB')(A'B)}{(AB)(A'B')}\\
&=  \frac{\mu_0 I}{2\pi}\ln\frac{D^2+d^2}{D^2}.
\end{align*}
Finally we can determine the mutual inductance by simply dividing by the current
\begin{align*}
M_{12}'&= \frac{\Lambda_{12}'}{I}\\
&= \frac{\mu_0}{2\pi}\ln\left(1+\frac{d^2}{D^2}\right).
\end{align*}
}

\answer{\begin{align*}
M^{'}_{12}=\frac{\mu_o}{2\pi}\ln \Big(1+\frac{d^2}{D^2}\Big)
\end{align*}}
\end{document}




































