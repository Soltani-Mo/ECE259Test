\documentclass[../../header.tex]{subfiles}

%%%Converted into Workbook 1-3
 
\begin{document}
\problem{For the structure composed of an infinitely long line charge distribution $\rho_l$ along the $z$-axis and a charged semi-cylinder with surface charge density $\rho_s$ at  $r=a$, $0 \leq \phi \leq \pi$, find the force per unit length on the semi-cylinder.}

\solution{
Let us define a coordinate system for the problem as follows:

\includegraphics{\ppath Q6.jpg} \\
Hence, the cylinder is expressed as: $r=a$, $\pi/2 \le \phi \le 3\pi/2, -\infty<z<\infty$. In class, we showed that the line charge distribution produces a field:
\begin{equation*}
\vect{E} = \frac{\rho_l}{2\pi \varepsilon_0 r} \uvect{a}_r
\end{equation*}

Hence, at the position of the cylinder, the field is:
\begin{equation*}
\vect{E} = \frac{\rho_l}{2\pi \varepsilon_0 a} \uvect{a}_r\,.
\end{equation*}

Now, consider a differential surface element on the surface of the cylinder:
$ds = r\,d\phi\,dz$, carrying charge $dQ = \rho_s a\,d\phi\,dz$ at position $\vect{R} = a \uvect{a}_r + z \uvect{a}_z$. Because of the field of the line charge, this $dQ$ receives a force:
\begin{equation*}
d\vect{F} = dQ \vect{E} = \rho_s a\,d\phi\,dz \frac{\rho_l}{2\pi \varepsilon_0 a} \uvect{a}_r = \frac{\rho_l \rho_s}{2\pi \varepsilon_0} d\phi\,dz\,(\uvect{a}_x \cos \phi + \uvect{a}_y \sin \phi ) 
\end{equation*}
Integrating to find the total force (assuming a length $L$ for the semi-cylinder):

\begin{equation*}
\vect{F} = \int_{\text{semi-cylinder}} d\vect{F} = \frac{\rho_l \rho_s}{2\pi \varepsilon_0} \int_{z=0}^{z=L} dz \int_{\phi = \pi/2}^{\phi = 3\pi/2}d\phi \left( \uvect{a}_x \cos\phi + \uvect{a}_y \sin \phi \right) = - \uvect{a}_x \frac{\rho_l \rho_s}{\pi \epsilon_0} L
\end{equation*}
Hence, per unit length:
\begin{equation*}
\frac{\vect{F}}{L} = - \uvect{a}_x \frac{\rho_l \rho_s}{\pi \varepsilon_0}
\end{equation*}
Confirm that the units are Newton/m.\\}
\answer{
\begin{equation*}
\frac{\vect{F}}{L} = - \uvect{a}_x \frac{\rho_l \rho_s}{\pi \varepsilon_0}
\end{equation*}}

\end{document}




































