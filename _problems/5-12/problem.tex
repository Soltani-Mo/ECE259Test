\documentclass[../../header.tex]{subfiles}

\begin{document}
% Same as Workbook 5-8
\problem{[Cheng P.5-9] Two lossy dielectric media with permittivities and conductivities ($\varepsilon_1,\sigma_1$) and ($\varepsilon_2,\sigma_2$) are in contact. An electric field with a magnitude $E_1$ is incident from medium 1 upon the interface at an angle $\alpha_1$ measured from the common normal, as in Fig. 5-10.
\begin{center}
\includegraphics[scale= 0.5]{\ppath cheng5-9_prompt.png}
\end{center}
\begin{enumerate}[label=(\alph*)]
\item Find the magnitude and direction of $\vect{E}_2$ in medium 2.
\item Find the surface charge density at the interface.
\item Compare the results in parts (a) and (b) with the case in which both media are perfect dielectrics.
\end{enumerate}

}

\solution{
%\begin{center}
%\includegraphics[scale= 0.9]{\ppath cheng5-9.pdf}
%\end{center}
\begin{enumerate}[label=(\alph*)]
\item 
\begin{align*}
\text{Eq. (3-118): } E_{1t} &= E_{2t} \rightarrow E_2 \sin \alpha_2 = E_1 \sin \alpha_1.\\
\text{Eq. (3-118): } J_{1n} &= J_{2n} \rightarrow \sigma_1 E_{1n} = \sigma_2 E_{2n} \rightarrow \sigma_2 E_2 \cos \alpha_2 = \sigma_1 E_1 \cos \alpha_1.\\
\end{align*}
Therefore,
\begin{align*}
E_2 &= E_1 \sqrt{\sin^2 \alpha_1 + (\frac{\sigma_1}{\sigma_2}\cos \alpha_1)^2}.\\
\tan \alpha_2 &= \frac{\sigma_2}{\sigma_1}\tan \alpha_1 \rightarrow \alpha_2 = \tan^{-1}(\frac{\sigma_2}{\sigma_1}\tan \alpha_1).
\end{align*}
\item
\begin{align*}
\text{Eq. (3-121b): } &D_{2n} - D_{1n} = \rho_s \rightarrow \varepsilon_2 E_{2n}-\varepsilon_1 E_{1n} = \rho_s\\
& \rho_s = (\frac{\sigma_1}{\sigma_2}\varepsilon_2 - \varepsilon_1)E_{1n} = (\frac{\sigma_1}{\sigma_2}\varepsilon_2 - \varepsilon_1) E_1 \cos \alpha_1.
\end{align*}
\item If both media are perfect dielectrics, $\sigma_1=\sigma_2=0$. $\alpha_2 = \tan^{-1}(\frac{\sigma_2}{\sigma_1}\tan \alpha_1)$ reverts to Eq (1-129), while $E_2 = E_1 \sqrt{\sin^2 \alpha_1 + (\frac{\sigma_1}{\sigma_2}\cos \alpha_1)^2}$ reverts to Eq. (3-130) and $\rho_s = 0$.
\end{enumerate}
}

\answer{\begin{enumerate}[label=(\alph*)]
\item \begin{align*}
\alpha_2=\tan^{-1}(\frac{\sigma_2}{\sigma_1}\tan\alpha_1)
\end{align*}
\item \begin{align*}
\rho_s=(\frac{\sigma_1}{\sigma_2}\varepsilon_2-\varepsilon_1)E_1\cos\alpha_1
\end{align*}
\item \begin{align*}
\sigma_1=\sigma_2\,\,and\,\,\rho_s=0
\end{align*}
\end{enumerate}}

\end{document}




































