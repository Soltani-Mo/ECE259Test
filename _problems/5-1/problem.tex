\documentclass[../../header.tex]{subfiles}

\begin{document}

\problem{[Cheng P.4-1] The upper and lower conducting plates of a large parallel-plate capacitor are separated by a distance $d$ and maintained at potentials $V_0$ and $0$, respectively. A dielectric slab of dielectric constant $6.0$ and uniform thickness $0.8d$ is placed over the lower plate. Assuming negligible fringing effect, determine
\begin{enumerate}[label=(\alph*)]
\item the potential and electric field distribution in the dielectric slab
\item the potential and electric field distribution in the air space between the dielectric slab and the upper plate,
\item the surface charge densities on the upper and lower plates.
\item Compare the results in part (b) with those without the dielectric slab.
\end{enumerate}
}

\solution{
%\begin{center}
%\includegraphics[scale= 0.6]{\ppath 4-1.png}
%\end{center}
Use subscripts d and a to denote dielectric and air regions respectively. $\nabla^2 V = 0$ in both regions.
\begin{align*}
\begin{split}
&V_d = c_1 y + c_2\\
&\vect{E}_d = -\uvect{a}_y c_1 \\
&\vect{D}_d = -\uvect{a}_y \varepsilon_0\varepsilon_r c_1 
\end{split}
\begin{split}
&V_a = c_3 y + c_4\\
&\vect{E}_a = -\uvect{a}_y c_3  \\
&\vect{D}_a = -\uvect{a}_y \varepsilon_0 c_3
\end{split}
\end{align*}
Boundary conditions:
\begin{itemize}
\item at $y=0$: $V_d=0$
\item at $y=d$: $V_a=V_0$
\item at $y=0.8d$: $V_d=V_a$, $\vect{D}_d=\vect{D}_a$ 
\end{itemize}
Solving these we arrive at
\begin{align*}
c_1 &= \frac{V_0}{(0.8+0.2\varepsilon_r)d}\\
c_2 &= 0\\
c_3 &= \frac{\varepsilon_r V_0}{(0.8+0.2\varepsilon_r)d}\\
c_4 &= \frac{(1-\varepsilon_r)V_0}{1+0.25\varepsilon_r}
\end{align*}
\begin{enumerate}[label=(\alph*)]
\item $V_d = \frac{5yV_0}{(4+\varepsilon_r)d}$, $\vect{E}_d = -\uvect{a}_y\frac{5V_0}{(4+\varepsilon_r)d}$.
\item $V_a = \frac{5\varepsilon_r y - 4(\varepsilon_r -1)d}{(4+\varepsilon_r)d}V_0$, $\vect{E}_a = -\uvect{a}_y\frac{5\varepsilon_rV_0}{(4+\varepsilon_r)d}$.
\item $(\rho_s)_{y=d}=-(D_a)_{y=d}=\frac{5\varepsilon_0\varepsilon_rV_0}{(4+\varepsilon_r)d}$.
$(\rho_s)_{y=0}=-(D_d)_{y=0}= -\frac{5 \varepsilon_0 \varepsilon_r V_0}{(4+\varepsilon_r)d}$.
\end{enumerate}
}

\answer{
%\begin{center}
%\includegraphics[clip,trim={0cm 0cm 0cm 8.5cm},scale= 0.6]{\ppath 4-1.png}
%\end{center}

\begin{enumerate}[label=(\alph*)]
\item $V_d = \frac{5yV_0}{(4+\varepsilon_r)d}$, $\vect{E}_d = -\uvect{a}_y\frac{5V_0}{(4+\varepsilon_r)d}$.
\item $V_a = \frac{5\varepsilon_r y - 4(\varepsilon_r -1)d}{(4+\varepsilon_r)d}V_0$, $\vect{E}_a = -\uvect{a}_y\frac{5\varepsilon_rV_0}{(4+\varepsilon_r)d}$.
\item $(\rho_s)_{y=d}=-(D_a)_{y=d}=\frac{5\varepsilon_0\varepsilon_rV_0}{(4+\varepsilon_r)d}$.
$(\rho_s)_{y=0}=-(D_d)_{y=0}= -\frac{5 \varepsilon_0 \varepsilon_r V_0}{(4+\varepsilon_r)d}$.
\end{enumerate}}
\end{document}




































