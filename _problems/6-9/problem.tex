\documentclass[../../header.tex]{subfiles}

\begin{document}

\problem{[Cheng P.6-13] A thin conducting wire is bent into the shape of a regular polygon of $N$ sides. A current $I$ flows in the wire. Show that the magnetic flux density at the center is 
\begin{align*}
\vect{B} = \uvect{a}_n \frac{\mu_0 N I }{2\pi b}\tan\frac{\pi}{N},
\end{align*}
where $b$ is the radius of the circle circumscribing the polygon and $\uvect{a}_n$ is a unit vector normal to the plane of the polygon. Show also that, as $N$ becomes very large, this result reduces to that given in Eq. (6-38) with z=0.
}

\solution{
Use Eq. (6-35) for a wire of length 2L.
\begin{align*}
\vect{B} = \uvect{a}_\phi \frac{\mu_0IL}{2\pi r \sqrt{L^2+r^2}}.
\end{align*}
Using the diagram below $\alpha = \frac{\pi}{N}$, $\frac{L}{r} = \tan \alpha = \tan \frac{\pi}{N}$.
\begin{center}
\includegraphics[scale = 0.5]{\ppath Cheng6-13_diagram.png}
\end{center}
Using this result
\begin{align*}
\vect{B} &= \uvect{a}_n N\left( \frac{\mu_0 IL}{2\pi r b}\right)\\
\vect{B} &= \uvect{a}_n \frac{N\mu_0 I}{2\pi b}\tan \frac{\pi}{N}.
\end{align*}
When $N$ is very large, $\tan \frac{\pi}{N}\approx \frac{\pi}{N}$. Therefore $\vect{B} \rightarrow \uvect{a}_n\frac{\mu_0I}{2b}$. This is the same as Eq. (6-38) with $z=0$.
}


\answer{Proof problem}
\end{document}




































