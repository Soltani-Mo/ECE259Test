\documentclass[../../header.tex]{subfiles}

\begin{document}

\problem{The electrostatic potential $V$ in a region is a function of a single rectangular coordinate $x$, $V(x)$ and is shown in the figure below. Sketch the components of the electric field intensity $\vect{E}$ in this region
\begin{center}
\includegraphics[scale = 0.6] {\ppath Q1.jpg}
\end{center}}

\solution{
The electric field intensity in the region is given by  $E_x(x) = -dV/dx$, i.e. it equals the negative of the derivative fo the function $V(x)$ at the coordinate $x$. In otherwords, $E_x(x)$ equals the negative of the slope of the $V(x)$ curve in the figure above at the corresponding absicssa point $x$, and based on this fact we sketch the function $E_x(x)$ as
\begin{center}
\includegraphics[clip,trim={15mm 12mm 15mm 28mm},width=0.6\textwidth]{\ppath Q1_solution.jpg}
\end{center}}
\answer{
\begin{center}
\includegraphics[clip,trim={15mm 12mm 15mm 28mm},width=0.6\textwidth]{\ppath Q1_solution.jpg}
\end{center}}

\end{document}




































