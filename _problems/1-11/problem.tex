
\documentclass[../../header.tex]{subfiles}

\begin{document}

\problem{A line charge of uniform density $\rho_l$ in free space forms a semicircle of radius $b$. Determine the magnitude and direction of the electric field intensity at the center of the semicircle.}

\solution{We can begin by placing the semicircle extending from $\phi=0$ to $\phi=\pi$ along the radius $r=b$. The semicircle is now within the $+y$ half of the $xy$ plane. From this, we can recognize by symmetry that the only electric field component that will be non-zero is in the $y$-direction. This is because any $\uvect{a}_x$ components of the field from one side of the semicircle are cancelled by an equal and opposite contribution from the other side.

Starting with the differential field from our contour of line charge to the origin,
\begin{align*}
	d\vect{E}&=-\dfrac{\rho_ldl}{4\pi\varepsilon_0 r^2}\uvect{a}_r
\end{align*}
and the fact that $\uvect{a}_y=b\sin\phi\uvect{a}_r$, we arrive at
\begin{align*}
	\vect{E}_y&=-\frac{1}{4\pi\epsilon_0}\int_0^{\pi}\frac{\rho_lbd\phi}{b^2}\sin\phi\vect{a}_y\\
	&=-\frac{\rho_l}{2\pi\epsilon_0b}\vect{a}_y
\end{align*}}

\answer{
\begin{align*}
	\vect{E}_y&=-\frac{\rho_l}{2\pi\epsilon_0b}\vect{a}_y
\end{align*}
}
\end{document}





































