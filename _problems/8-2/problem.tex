\documentclass[../../header.tex]{subfiles}

\begin{document}

\problem{[Cheng P.6-32] Consider a plane boundary ($y=0$) between air (region 1, $\mu_{r1} = 1$) and iron (region 2, $\mu_{r2} = 5000$).
\begin{enumerate}[label=(\alph*)]
\item Assuming $\vect{B}_1 = \uvect{a}_x0.5-\uvect{a}_y10$(mT), find $\vect{B}_2$ and the angle $\vect{B}_2$ makes with the interface.
\item Assuming $\vect{B}_2 = \uvect{a}_x10+\uvect{a}_y0.5$(mT), find $\vect{B}_1$ and the angle $\vect{B}_1$ makes with the normal to the interface.
\end{enumerate}
}

\solution{The problem has the structure shown in the diagram below.
\begin{center}
\includegraphics[scale =1]{\ppath Cheng6-32_diagram.png}
\end{center}
\begin{enumerate}[label=(\alph*)]
\item The magnetic flux densities in both regions are given by
\begin{align*}
\vect{B}_1 &= \uvect{a}_x0.5-\uvect{a}_y10\; \text{(mT)}\\
\vect{B}_2 &= \uvect{a}_xB_{2x}-\uvect{a}_yB_{2y}.
\end{align*}
From the boundary conditions we have
\begin{align*}
H_{1x} &= H_{2x} \rightarrow \frac{0.5}{\mu_0}=\frac{B_{2x}}{5000\mu_0}\\
B_{2x} &= 2500\; \text{(mT)}\\
B_{1y} &= B_{2y}\\
B_{2y} &= -10\; \text{(mT)}
\end{align*}
Putting them together we get
\begin{align*}
\vect{B}_2 &= \uvect{a}_x2500-\uvect{a}_y10\; \text{(mT)}
\end{align*}
To solve for the angle we use the first boundary condition $H_{1x} = H_{2x} \rightarrow \mu_2 B_{1x} = \mu_1 B_{2x}$. Now if we divide both sides by their $y$-components (which are equal from the other set of boundary conditions) we get $\frac{B_{1x}}{B_{1y}}=\frac{\mu_1}{\mu_2}\frac{B_{2x}}{B_{2y}}$. Finally using $\tan \alpha = \frac{B_x}{B_y}$ we get
\begin{align*}
\tan \alpha_2 &= \frac{\mu_2}{\mu_1}\tan \alpha_1 = 5000 \frac{B_{1x}}{B_{1y}} = 250\\
\alpha_2 &= 89.8^\circ, \; \alpha_2' = 0.2^\circ .
\end{align*}
\item The magnetic flux densities in both regions are given by
\begin{align*}
\vect{B}_2 &= \uvect{a}_x10+\uvect{a}_y0.5\; \text{(mT)}\\
\vect{B}_1 &= \uvect{a}_xB_{1x}+\uvect{a}_yB_{1y}.
\end{align*}
As before, from the boundary conditions we have
\begin{align*}
H_{1x} &= H_{2x} \rightarrow \frac{B_{1x}}{\mu_0}=\frac{10}{5000\mu_0}\\
B_{1x} &= 0.002\; \text{(mT)}\\
B_{1y} &= B_{2y}\\
B_{2y} &= 0.5\; \text{(mT)}
\end{align*}
Putting them together we get
\begin{align*}
\vect{B}_2 &= \uvect{a}0.002+\uvect{a}_y0.5\; \text{(mT)}
\end{align*}
To find the angle we use the same process as part (a)
\begin{align*}
\alpha_1 = \tan^{-1} \frac{B_{1x}}{B_{1y}} \approx \frac{0.002}{0.5} = 0.23^\circ
\end{align*}

\end{enumerate}
}


\answer{\begin{enumerate}[label=(\alph*)]
\item \begin{align*}
\vect{B_2}&=a_x 2500 - a_y 10\,\,\text{(mT)}\\
\alpha_2&=89.8^o
\end{align*}
\item \begin{align*}
\vect{B_1}&=a_x 0.002 + a_y 0.5\,\,\text{(mT)}\\
\alpha_1&=0.23^o
\end{align*}
\end{enumerate}}
\end{document}




































