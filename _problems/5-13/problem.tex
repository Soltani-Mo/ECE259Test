\documentclass[../../header.tex]{subfiles}

\begin{document}

\problem{[Cheng P.5-14] Refer to the flat conducting quarter-circular washer in Example 5-6 and Fig 5-8. Find the resistance between the curved sides.
\begin{center}
\includegraphics[scale= 0.6]{\ppath cheng5-14_prompt.png}
\end{center}
}

\solution{
%\begin{center}
%\includegraphics[scale= 0.6]{\ppath 5-14.png}
%\end{center}
Starting with Poisson's equation
\begin{align*}
\nabla^2V = 0 \rightarrow \frac{1}{r}\frac{\partial}{\partial r}\left( r \frac{\partial V}{\partial r}\right) = 0.
\end{align*}
This has the solution
\begin{align*}
V(r) = c_1 \ln r + c_2.
\end{align*}
To solve for $c_1$ and $c_2$, we use the boundary conditions $V(a) = V_0$ and $V(b) = 0$. This yields 
\begin{align*}
V(r) = V_0 \frac{\ln (b/r)}{\ln (b/a)}.
\end{align*}
From this we can solve for the current $I$
\begin{align*}
\vect{E}(r) &= -\uvect{a}_r \frac{\partial V}{\partial r} = \uvect{a}_r\frac{V_0}{r\ln (b/a)}.\\
\vect{J}(r) &= \sigma \vect{E}_r.\\
I &= \int_S \vect{J}\cdot dS = \int_0^{\pi/2}\vect{J}\cdot (\uvect{a}_r hrd\phi) \\
  &= \frac{\pi\sigma h V_0}{2\ln (b/a)}.
\end{align*}
Now with Ohm's law we can solve for the resistance $R$
\begin{align*}
R = \frac{V_0}{I} = \frac{2 \ln (b/a)}{\pi \sigma h}.
\end{align*}
}

\answer{
\begin{align*}
R=\frac{2\ln(b/a)}{\pi\sigma h}
\end{align*}}
\end{document}




































