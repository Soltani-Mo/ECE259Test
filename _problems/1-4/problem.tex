
\documentclass[../../header.tex]{subfiles}

%%%Converted into Workbook 1-2

\begin{document}
\problem{Show that for large $z$, the electric field created on the $z$-axis (observation point $(0,0,z)$) by a semi-circular line charge with density $\rho_l$ at $z=0$, $r=a$ and $0 \leq \phi <\pi$, 
is equivalent to the field of a point charge with the same amount of charge, located at the origin.}

\solution{
The total charge on the line is: $Q = \rho_l \pi a$. If that were considered as a point charge at the origin, it would produce at $(0,0,z)$ an electric field intensity:
\begin{equation*}
\vec{E} = \frac{\rho_l \pi a}{4\pi \varepsilon_o z^2} \vec{a}_z
\end{equation*}
For the semi-circular charge distribution, we apply the superposition formula:
\begin{equation*}
\vec{E} = \int_\text{semi-circle} \frac{dQ'}{4\pi |\vec{R} - \vec{R'}|^3} \left( \vec{R} - \vec{R'} \right)
\end{equation*}
We follow the steps we saw in class to calculate this integral. 

1. Choose the coordinate system: Although the problem does not present any symmetry as other problems we saw in class, its charge distribution can be easily described in terms of cylindrical coordinates. In particular, the semi-circle  is  expressed as:
\begin{equation*}
r = a, \quad -\frac{\pi}{2} \le \phi \le \frac{\pi}{2}, \quad z = 0.
\end{equation*}
(Compare this with the Cartesian system; how would it be expressed in Cartesian coordinate?). Hence, we choose cylindrical system.

2. Find $dQ'$. Here $dQ' = \rho_l\,dl'$, where $dl' = a\,d\phi'$, a differentially small arc-length on the semi-circle. Note the use of primed coordinate for the source points.

3. Find the vectors $\vec{R}$, $\vec{R'}$, etc:
\begin{itemize}
	\item $\vec{R}$ is the position vector of the observation point: $\vec{R} = z\vec{a}_z$.
	\item $\vec{R'}$ is the position of $dQ'$, $\vec{R'} = a \vec{a}_r$. Whenever this is expressed in terms of non-cartesian unit vectors (like here), express all these vectors in terms of the Cartesian unit vectors. You will see why in a moment. Here:
\begin{equation*}
\vec{R'} = a \vec{a}_{r'} = a \left( \cos \phi \vec{a}_x + \sin \phi \vec{a}_y \right)
\end{equation*}
Note how this vector depends on $\phi'$ (i.e. it is different at different points at the semi-circle). Then:
\begin{equation*}
\vec{R} - \vec{R'} = z \vec{a}_z - a \cos \phi' \vec{a}_x - a \sin \phi' \vec{a}_y
\end{equation*}
and
\begin{equation*}
|\vec{R} -\vec{R'}| = \sqrt{z^2 + a^2}
\end{equation*}
(as is obvious from the figure, the distance from any point on the semi-circle to $(0,0,z)$ is the same).
\end{itemize}

Now we have all the components of the field formula. Mere back-substitution results in:
\begin{eqnarray*}
\vec{E} &=& \int_\text{semi-circle} \frac{\rho_l a\,d\phi'}{4\pi \varepsilon_o (z^2 + a^2)^{3/2}} 
(z \vec{a}_z - a \cos \phi' \vec{a}_x - a \sin \phi' \vec{a}_y )\\
&=& \frac{\rho_l a}{4\pi \varepsilon_o (z^2 + a^2)^{3/2} }\left(
z\vec{a_z} \int_{-\pi/2}^{\pi/2} d\phi' - a\vec{a}_x \int_{-\pi/2}^{\pi/2}\cos\phi' d\phi' - a \vec{a}_y \int_{-\pi/2}^{\pi/2} \sin \phi' d\phi'
\right)\\
&=& \frac{\rho_l a}{4\pi \varepsilon_o (z^2 + a^2)^{3/2} } \left( \pi z \vec{a}_z - 2a \vec{a}_x \right)
\end{eqnarray*}

Note how the introduction of the Cartesian unit vectors instead of the $r'$-unit vector clarified the variables of the integration. Had we not done that, the dependence of $\vec{a}_{r'}$ on $\phi'$ would have remained implicit. Many times people tend to forget this dependence and derive totally different results.

But, for $|z| \gg a$, the first term is much larger than the second, and also, $z^2 + a^2 \approx z^2$. Hence,
\begin{equation*}
\vec{E} = \frac{\rho_l\,a}{4\pi \varepsilon_o (z^2)^{3/2} } \pi z \vec{a}_z = \frac{\rho_l a \pi}{4\pi \varepsilon_o z^2} \vec{a}_z
\end{equation*}
which indeed is the field of a point charge at the origin carrying the whole charge of the semi-circle.}
\answer{
Proof Problem}
\end{document}




































