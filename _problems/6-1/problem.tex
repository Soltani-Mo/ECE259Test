\documentclass[../../header.tex]{subfiles}

\begin{document}

\problem{[Cheng P.6-1] A positive point charge $q$ of mass $m$ is injected with a velocity $\textbf{u}_0=\textbf{a}_yu_0$ into the $y>0$ region where a uniform magnetic field $\textbf{B}=\textbf{a}_xB_0$ exists. Obtain the equation of motion of the charge, and describe the path that the charge follows.}

\solution{The force on the point charge due to its velocity and the magnetic field will be $\textbf{F}=m\textbf{a}=q\textbf{u}\times\textbf{B}$. This gives two coupled differential equations
	\begin{align*}
		\begin{cases} 
			\frac{du_y}{dt}&=\ \ \ \omega_ou_z\\
			\frac{du_z}{dt}&=-\omega_ou_y
		\end{cases}
	\end{align*}
	These equations can be combined to get
	\begin{align*}
		\frac{d^2u_z}{dt^2}&=-\omega_0^2u_z\\
		u_z&=A\cos(\omega_0t)+B\sin(\omega_0t)
	\end{align*}
Applying the boundary condition that at $t=0$, $u_z=0$ gives $A=0$ so
	\begin{align*}
		u_z=B\sin(\omega_0t). 
	\end{align*}
Substituting this into the second of the coupled differential equations,
	\begin{align*}
		u_y=-B\cos(\omega_0t)
	\end{align*}
Applying the boundary condition that at $t=0$, $u_y=u_0$ gives $B=-u_0$ so
	\begin{align*}
		\begin{cases} 
			u_y&=\ \ \ u_0\cos(\omega_0t)\\
			u_z&=-u_0\sin(\omega_0t)
		\end{cases}
	\end{align*}
Integrating these equations and applying the boundary condition that at $t=0$, $y=z=0$ gives the equations of motion for the charge
	\begin{align*}
		\begin{cases} 
			y&=\frac{u_0}{\omega_0}\sin(\omega_0t)\\
			z&=\frac{u_0}{\omega_0}\cos(\omega_0t)-\frac{u_0}{\omega_0}
		\end{cases}
	\end{align*}
These equations can be combined into a single equation of motion
	\begin{align*}
		y^2+\Big(z+\frac{u_0}{\omega_0}\Big)^2=\Big(\frac{u_0}{\omega_0}\Big)^2
	\end{align*}
which is the equation of a circle in the $yz$ plane of radius $u_0/\omega_0$ centred at $y=0$, $z=-u_0/\omega_0$.}
\answer{
\begin{align*}
		y^2+\Big(z+\frac{u_0}{\omega_0}\Big)^2=\Big(\frac{u_0}{\omega_0}\Big)^2
	\end{align*}
	}
\end{document}




































