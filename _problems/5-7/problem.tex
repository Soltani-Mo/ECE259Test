\documentclass[../../header.tex]{subfiles}

\begin{document}

\problem{[Cheng P.4-7] A point charge $Q$ exists at a distance $d$ above a large grounded conducting plane. Determine
\begin{enumerate}[label=(\alph*)]
\item the surface charge density $\rho_s$,
\item the total charge induced on the conducting plane.
\end{enumerate}
}

\solution{
%\begin{center}
%\includegraphics[scale= 0.6]{\ppath 4-7.png}
%\end{center}
\begin{center}
\includegraphics[scale= 1]{\ppath 4-7_diagram.png}
\end{center}
With reference to the diagram above,
\begin{align*}
\vect{E}\mid_{y=0} &= -\uvect{a}_y \frac{Q}{4\pi\varepsilon R^2}(2 \sin \theta)\\
&= -\uvect{a}_y\frac{Qd}{2\pi\varepsilon(d^2+r^2)^{3/2}}.
\end{align*}
Now we can find the surface charge density $\rho_s$ and total charge $Q_{\text{tot}}$ on the conducting plate.
\begin{enumerate}[label=(\alph*)]
\item \begin{align*}
\rho_s &= \uvect{a}_y\cdot \varepsilon \vect{E}\mid_{y=0}\\
\rho_s &= -\frac{Qd}{2\pi(d^2+r^2)^\frac{3}{2}}
\end{align*}
\item \begin{align*}
Q_{\text{tot}} &= \int_0^{\infty}\rho_s 2\pi r dr \\
&= -Q
\end{align*}
\end{enumerate}
}

\answer{
\begin{enumerate}[label=(\alph*)]
\item \begin{align*}
\rho_s=-\frac{Qd}{2\pi(d^2+r^2)^\frac{3}{2}}
\end{align*}
\item $ Q_{\text{tot}} = -Q$
\end{enumerate}
}

\end{document}




































