\documentclass[../../header.tex]{subfiles}

\begin{document}
% Same as Workbook 6-5
\problem{[Cheng P.6-12] Two identical coaxial coils, each of $N$ turns and radius $b$, are separated by a distance $d$, as depicted in Fig. 6-39. A current $I$ flows in each coil in the same direction. 
\begin{enumerate}[label=(\alph*)]
\item Find the magnetic flux density $\vect{B}=\uvect{a}_x B_x$ at a point midway between the coils. 
\item Show that $dB_x/dx$ vanishes at the midpoint.
\item Find the relation between $b$ and $d$ such that $d^2B_x/dx^2$ also vanishes at the midpoint. 
\end{enumerate}
Such a pair of coils are used to obtain an approximately uniform magnetic field in the midpoint region. They are known as \textbf{\textit{Helmholtz coils}}.
\begin{center}
\includegraphics[scale= 0.7]{\ppath cheng6-12_prompt.png}
\end{center}
}

\solution{
\begin{center}
\includegraphics[scale= 0.9]{\ppath cheng6-12_diagram.png}
\end{center}
Using Eq. (6-38) and the diagram above
\begin{align*}
B_x = \frac{N\mu_0Ib^2}{2}\left\{ \frac{1}{[(d/2+x)^2 + b^2]^{3/2}}-\frac{1}{[(d/2-x)^2 + b^2]^{3/2}} \right\}.
\end{align*}
\begin{enumerate}[label=(\alph*)]
\item At $x=0$,
\begin{align*}
B_x = \frac{N\mu_0Ib^2}{[(d/2)^2+b]^{3/2}}.
\end{align*}
\item 
\begin{align*}
\frac{dB_x}{dx} = \frac{N\mu_0Ib^2}{2}\left\{ -\frac{3(d/2+x)}{[(d/2+x)^2 + b^2]^{5/2}}+\frac{3(d/2-x)}{[(d/2-x)^2 + b^2]^{5/2}} \right\}.
\end{align*}
At the midpoint $x=0$, $\frac{dB_x}{dx}=0$.
\item
\begin{align*}
\begin{split}
\frac{dB_x^2}{dx^2} =& -\frac{3N\mu_0Ib^2}{2}\left\{ \frac{1}{[(d/2+x)^2 + b^2]^{5/2}} + \frac{1}{[(d/2-x)^2 + b^2]^{5/2}} \right. \\ & \left. - \frac{5(d/2+x)^2}{[(d/2+x)^2 + b^2]^{7/2}} - \frac{5(d/2-x)^2}{[(d/2-x)^2 + b^2]^{7/2}} \right\}.
\end{split}
\end{align*}
At $x=0$,
\begin{align*}
\frac{dB_x^2}{dx^2} = -3N\mu_0Ib^2 \left\{ \frac{b^2-4(d/2)^2}{[(d/2)^2+b^2]^{7/2}} \right\}.
\end{align*}
This equals $0$ if $b=d$.
\end{enumerate}
}

\answer{\begin{enumerate}[label=(\alph*)]
\item \begin{align*}
B_x = \frac{N\mu_0Ib^2}{[(d/2)^2+b]^{3/2}}
\end{align*}
\item Proof problem
\item Proof problem, $b=d$
\end{enumerate}
}
\end{document}




































