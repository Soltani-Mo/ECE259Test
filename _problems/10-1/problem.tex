\documentclass[../../header.tex]{subfiles}

\begin{document}

\problem{[Cheng P.7-7] A conducting sliding bar oscillates over two parallel conducting rails in a sinusoidally varying magnetic field 
\begin{align*}
\vect{B} = \uvect{a}_z 5\cos\omega t \; \text{(mT)},
\end{align*}
as shown in Fig. 7-13. The position of the sliding bar is given by $x=0.35(1-\cos \omega t)$ (m), and the rails are terminated in a resistance $R=0.2$ ($\Omega$). Find $i$ in Fig.7-13.
\begin{center}
\includegraphics[scale = 0.45]{\ppath Cheng7-7_prompt.png}
\end{center}
}

\solution{ The flux through the surface composed of the rails, resistor, and bar as a function of time is given by 
\begin{align*}
\Phi(t) &= \vect{B}(t)\cdot \vect{S}(t)\\
&= -(5 \cos\omega t)\cdot 0.2(0.7-x)\\
&= -0.35\cos \omega t (1+\cos \omega t)\; \text{(mT)}.
\end{align*}
We can now find the current $i$
\begin{align*}
i &= \frac{v}{R} = \frac{1}{R}\cdot -\frac{d\Phi}{dt}\\
&= -\frac{1}{R} 0.35 \omega(\sin\omega t + \sin 2\omega t)\\
&= -1.75\omega \sin\omega t(1+2\cos\omega t)\; \text{(mA)}.
\end{align*}
}
\answer{\begin{align*}
i=-1.75\omega \sin\omega t(1+2\cos\omega t)\; \text{(mA)}.
\end{align*}}
\end{document}




































