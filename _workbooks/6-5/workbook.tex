\documentclass[../../header.tex]{subfiles}

\begin{document}
%Cheng 6-12. 
\textbf{Goal:} Two identical coaxial coils, each of $N$ turns and radius $b$, are separated by a distance $d$, as depicted in Fig.~Q6-39. A current $I$ flows in each coil in the same direction. Find the magnetic flux density midway between the coils, and show that $\frac{d B_x}{dx}$ vanishes at the midpoint. Find the relation between $b$ and $d$ such that $\frac{d^2 B_x}{dx^2}$ also vanishes at the midpoint. Such a pair of coils are used to obtain an approximately uniform magnetic field in the midpoint region. They are known as Helmholtz coils.
\begin{center}
\includegraphics[width=0.3\textwidth]{\wpath Cheng6-12.png}\\
Fig.~6-39
\end{center}
\textbf{Steps:} 
\begin{enumerate}
\item Find the magnetic flux density $\vec{B} = \vec{a}_x B_x$ at a point midway between the coils.\\
\solution{
Use the equation
\begin{equation*}
B_x = \frac{N \mu_o I b^2}{2} \left( \frac{1}{ ((d/2 + x)^2 + b^2)^{3/2}} + \frac{1}{ ((d/2 - x)^2 + b^2)^{3/2}} \right)
\end{equation*}
Then at the midpoint
\begin{equation*}
B_x = \frac{N \mu_o I b^2}{ ( (d/2)^2 + b^2)^{3/2} }
\end{equation*}
}

\item Show that $\frac{d B_x}{dx}$ vanishes at the midpoint.\\
\solution{
\begin{equation*}
\frac{d B_x}{d x} = \frac{N \mu_o I b^2}{2} \left( - \frac{3 (d/2 + x)}{( (d/2+x)^2 + b^2)^{5/2}} + \frac{3 (d/2 - x)}{( (d/2-x)^2 + b^2)^{5/2}} \right)
\end{equation*}
Clearly, the equation is zero for $x = 0$.
}

\item Find the relation between $b$ and $d$ such that $\frac{d^2 B_x}{dx^2}$ also vanishes at the midpoint.\\
\solution{
\begin{eqnarray*}
\frac{d^2 B_x}{dx^2} = -\frac{3N\mu_oIb^2}{2} &&\left(
\frac{1}{( (d/2+x)^2 + b^2 )^{5/2}} - \frac{5 (d/2+x)^2}{( (d/2+x)^2 + b^2 )^{7/2}} + \right.\\
&&\left. \frac{1}{( (d/2-x)^2 + b^2 )^{5/2}} -  \frac{5 (d/2-x)^2}{( (d/2-x)^2 + b^2 )^{7/2}}
\right)
\end{eqnarray*}
Now at $x = 0$
\begin{equation*}
\frac{d^2 B_x}{dx^2} = -3N\mu_oIb^2 \left( \frac{ b^2 - 4(d/2)^2}{( (d/2)^2 + b^2)^{7/2}} \right)
\end{equation*}
The above equation is zero if $b = d$.
}
\end{enumerate}
\answer{\begin{enumerate}[label=(\alph*)]
\item \begin{equation*}
B_x = \frac{N \mu_o I b^2}{2} \left( \frac{1}{ ((d/2 + x)^2 + b^2)^{3/2}} + \frac{1}{ ((d/2 - x)^2 + b^2)^{3/2}} \right)
\end{equation*}
\item Proof problem
\item b\,=\,d
\end{enumerate}}
\end{document}


