\documentclass[../../header.tex]{subfiles}

\begin{document}
%Cheng 5-9. 
\textbf{Goal:} Two lossy dielectric media with permittivities and conductivities ($\varepsilon_1, \sigma_1$) and ($\varepsilon_2, \sigma_2$) are in contact. An electric field with a magnitude $E_1$ is incident from medium 1 upon the interface at an angle $\alpha_1$ measured from the common normal, as in Fig.~5-10. Find the electric field in medium 2 $E_2$ and the surface charge density. How would these change if the media were perfect dielectrics?
\begin{center}
\includegraphics[width=0.3\textwidth]{\wpath Cheng5-9.pdf}
\end{center}
\textbf{Steps:} 
\begin{enumerate}
\item State the boundary conditions for the E-field, D-field and current density across the boundary.\\
\solution{
\begin{eqnarray*}
E_{1t} &=& E_{2t}\\
J_{1n} &=& J_{2n}\\
D_{1n} - D_{2n} &=& \rho_s
\end{eqnarray*}
}

\item Find the magnitude and direction of $\vec{E}_2$ in medium 2.\\
\solution{
From the boundary conditions for E-field and current density:
\begin{eqnarray*}
E_2 \sin\alpha_2 &=& E_1\sin\alpha_1\\
\sigma_2 E_2 \cos\alpha_2 &=& \sigma_1 E_1 \cos\alpha_1\\
E_2 &=& E_1 \sqrt{\sin^2\alpha_1 + \left(\frac{\sigma_1}{\sigma_2} \cos\alpha_1\right)^2} \quad (A)\\
\alpha_2 &=& \tan^{-1} \left( \frac{\sigma_2}{\sigma_1} \tan \alpha_1 \right) \quad \quad \quad (B)
\end{eqnarray*}
}

\item Find the surface charge density at the interface.\\
\solution{
\begin{eqnarray*}
\varepsilon_2 E_{2n} - \varepsilon_1 E_{1n} &=& \rho_s\\
\rho_s &=& \left( \frac{\sigma_1}{\sigma_2}\varepsilon_2 - \varepsilon_1\right) E_{1n} = \left(\frac{\sigma_1}{\sigma_2}\varepsilon_2 - \varepsilon_1\right) E_1 \cos\alpha_1
\end{eqnarray*}
}

\item Compare the results in parts (a) and (b) with the case in which both media are perfect dielectrics.\\
\solution{
If both media are perfect dielectrics, then $\sigma_1 = \sigma_2 = 0$ and Eq.~A becomes
\begin{equation*}
E_2 = E_1 \sqrt{\sin^2\alpha_1 + \left(\frac{\varepsilon_1}{\varepsilon_2} \cos\alpha_1\right)^2}
\end{equation*}
and Eq.~B becomes
\begin{equation*}
\alpha_2 = \tan^{-1} \left( \frac{\varepsilon_2}{\varepsilon_1} \tan \alpha_1 \right)
\end{equation*}
}

\end{enumerate}
\answer{
\begin{eqnarray*}
E_2 &=& E_1 \sqrt{\sin^2\alpha_1 + \left(\frac{\sigma_1}{\sigma_2} \cos\alpha_1\right)^2} \\
\alpha_2 &=& \tan^{-1} \left( \frac{\sigma_2}{\sigma_1} \tan \alpha_1 \right)
\end{eqnarray*}
\begin{eqnarray*}
\rho_s &=& \left(\frac{\sigma_1}{\sigma_2}\varepsilon_2 - \varepsilon_1\right) E_1 \cos\alpha_1
\end{eqnarray*}
For perfect dielectrics:\\
\begin{equation*}
E_2 = E_1 \sqrt{\sin^2\alpha_1 + \left(\frac{\varepsilon_1}{\varepsilon_2} \cos\alpha_1\right)^2}
\end{equation*}
\begin{equation*}
\alpha_2 = \tan^{-1} \left( \frac{\varepsilon_2}{\varepsilon_1} \tan \alpha_1 \right)
\end{equation*}
}
\end{document}






