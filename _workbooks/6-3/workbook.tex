\documentclass[../../header.tex]{subfiles}

\begin{document}
%Cheng 6-5: 
\textbf{Goal:} A current $I$ flows in a $w\times w$ square loop as shown in Fig. a) below. Find the magnetic flux density at the off-center point $P(w/4, w/2)$.

\begin{center}
\includegraphics[scale = 0.6]{\wpath workbook5.png}
\end{center}
\textbf{Steps:} 
\begin{enumerate}
\item The total magnetic flux density at point $P(w/4,w/2)$ is the sum of the contributions from each segment of the loop. Firstly, we consider the magnetic flux density at an arbitrary point $Q(b,a)$ due to a line current source of length $w$, as shown in Fig b). What is the source position vector $\vect{R}'$ for this configuration?\\
\solution{
\begin{align*}
\vect{R}' = y' \uvect{a}_y
\end{align*}
}

\item Determine the observation vector $\vect{R}$.\\
\solution{
\begin{align*}
\vect{R} = b \uvect{a}_x + a \uvect{a}_y
\end{align*}
}

\item Determine the differential length vector ${\operatorname d}\vect{l}'$.\\
\solution{
\begin{align*}
d\vect{l}' = dy' \uvect{a}_y
\end{align*}
}

\item Determine the differential magnetic flux density.\\
\solution{
\begin{align*}
\vect{R} - \vect{R}' &= b \uvect{a}_x + a \uvect{a}_y \\
d\vect{B} &= \frac{\mu_0 I}{4 \pi} \frac{d\vect{l}' \times \left(\vect{R} - \vect{R}' \right)}{ \abs{\vect{R} - \vect{R}'}^3} \\
&= \frac{\mu_0 I}{4\pi} \frac{dy' \uvect{a}_y \times (b\uvect{a}_x + (a-y')\uvect{a}_y}{\left(b^2 + (a-y')^2 \right)^{3/2}} \\
&= -\uvect{a}_z \frac{\mu_0 I}{4\pi } \frac{b dy'}{\left( b^2 + (a-y')^2 \right)^{3/2}}
\end{align*}
}

\item Integrate.\\
\solution{
\begin{align*}
\vect{B} &= -\uvect{a}_z \frac{\mu_0 I}{4\pi} \int_0^{w} \frac{b}{\left( b^2 + (a-y')^2 \right)^{3/2}} dy'\\
&= -\uvect{a}_z \frac{\mu_0 I}{4\pi b} \left[ \frac{w-a}{\left((a-w)^2 + b^2 \right)^{1/2}} + \frac{a}{\left(a^2 + b^2\right)^{1/2}} \right]
\end{align*}
}

\item Use the result in part 6) to compute magnetic flux density at point $P$ due to:\\
\solution{
\begin{itemize}
\item segment A (as marked in Fig. a):
%\vskip 24pt
For $a = 3/4 w$, and $b = w/2$
\begin{align*}
\vect{B}_A = -\uvect{a}_z \frac{\mu_0 I}{\pi w} \left[ 0.6396 \right]
\end{align*}
\item segment B :
For $a = w/2$ and $b = w/4$
\begin{align*}
\vect{B}_B = -\uvect{a}_z \frac{\mu_0 I}{\pi w} \left[ 1.7889 \right]
\end{align*}
%\vskip 24pt
\item segment C:
%\vskip 24pt
For $a = w/4$ and $b=w/2$
\begin{align*}
\vect{B}_C = -\uvect{a}_z \frac{\mu_0 I}{\pi w} \left[ 0.6397 \right]
\end{align*}
\item segment D:
%\vskip 24pt
For $a = w/2$ and $b = 3w/4$
\begin{align*}
\vect{B}_D = -\uvect{a}_z \frac{\mu_0 I}{\pi w} \left[ 0.3698 \right]
\end{align*}
\end{itemize}\,.
}

\item What is the total magnetic flux density at point $P(w/4, w/2)$?\\
\solution{
\begin{align*}
\vect{B} &= \vect{B}_A + \vect{B}_B + \vect{B}_C + \vect{B}_D\\
&= -\uvect{a}_z \frac{\mu_0 I}{\pi w} \left[ 3.44 \right]
\end{align*}
}
\end{enumerate}
\answer{\begin{align*}
\vect{B} = -\uvect{a}_z \frac{\mu_0 I}{\pi w} \left[ 3.44 \right]
\end{align*}}


\end{document}

