\documentclass[../../header.tex]{subfiles}

\begin{document}
%Cheng 6-22. 
\textbf{Goal:} A circular rod of magnetic material with permeability $\mu$ is inserted coaxially in the long solenoid of Fig.~6-4. The radius of the rod, $a$, is less than the inner radius, $b$, of the solenoid. The solenoid has $n$ turns per unit length and carries a current $I$. Find $\vec{H}$, $\vec{B}$, and $\vec{M}$ inside the solenoid, as well as current densities $\vec{J}_m$ and $\vec{J}_{m,s}$.
\begin{center}
\includegraphics[width=0.8\textwidth]{\wpath Cheng6-22.jpg}\\
Fig.~6-4. A long solenoid with closely wound windings carrying a current $I$.
\end{center}
\textbf{Steps:} 
\begin{enumerate}
\item What is the $\vec{H}$-field inside the solenoid?\\
\solution{
%\vskip 100pt
\begin{eqnarray*}
\vec{H} &=& \vec{a}_z n I
\end{eqnarray*}
}

\item What is the $\vec{B}$-field inside the solenoid?\\
%\vskip 100pt
\solution{
\begin{eqnarray*}
\vec{B} &=& \vec{a}_z\mu n I \quad \text{for} \quad r < a\\
\vec{B} &=& \vec{a}_z\mu_o n I \quad \text{for} \quad a < r < b
\end{eqnarray*}
}

\item What is the $\vec{M}$ inside the solenoid?\\
%\vskip 100pt
\solution{
\begin{equation*}
\vec{M} = \frac{\vec{B}}{\mu_o} - \vec{H}
\end{equation*}
For $r < a$,
\begin{equation*}
\vec{M} = \vec{a}_z \left( \frac{\mu}{\mu_o} - 1 \right) nI
\end{equation*}
For $a < r < b$,
\begin{equation*}
\vec{M} = 0
\end{equation*}
}

\item Calculate the current densities $\vec{J}_m$ and $\vec{J}_{m,s}$ equivalent to magnetization.\\
%\vskip 100pt
\solution{
\begin{equation*}
\vec{J}_m = \nabla \times \vec{M} = 0
\end{equation*}
\begin{equation*}
\vec{J}_{ms} = \vec{M} \times \vec{a}_n = (\vec{a}_z \times \vec{a}_r) \left(\frac{\mu}{\mu_o}-1\right) n I = \vec{a}_\phi \left( \frac{\mu}{\mu_o} -1 \right) n I
\end{equation*}
}
\end{enumerate}
\answer{
\begin{eqnarray*}
\vec{H} &=& \vec{a}_z n I
\end{eqnarray*}
\begin{eqnarray*}
\vec{B} &=& \vec{a}_z\mu n I \quad \text{for} \quad r < a\\
\vec{B} &=& \vec{a}_z\mu_o n I \quad \text{for} \quad a < r < b
\end{eqnarray*}
For $r < a$,
\begin{equation*}
\vec{M} = \vec{a}_z \left( \frac{\mu}{\mu_o} - 1 \right) nI
\end{equation*}
For $a < r < b$,
\begin{equation*}
\vec{M} = 0
\end{equation*}
\begin{equation*}
\vec{J}_{ms} = \vec{a}_\phi \left( \frac{\mu}{\mu_o} -1 \right) n I
\end{equation*}
}
\end{document}




































