\documentclass[../../header.tex]{subfiles}

\begin{document}
%Notaros 5-3
\textbf{Goal:} \textit{Uniformly magnetized square ferromagnetic plate.} A uniformly magnetized square ferromagnetic plate of side length $a$ and thickness $d$ ($d \ll a$) is situated in air. With reference to the coordinate system in Fig.~5.36, the magnetization vector in the plate is given by $\vec{M} = M_o \vec{a}_z$, where $M_o$ is a constant. Determine the magnetic flux density vector at an arbitrary point on the $z$ axis.
\begin{center}
\includegraphics[width=0.5\textwidth]{\wpath Q5-3.png}
\end{center}
\textbf{Steps:} 
\begin{enumerate}
\item Determine the current densities $\vect{J}_m$ and $\vect{J}_{m,s}$ equivalent to magnetization, and sketch them in Fig.~5.36.\\
%\vskip 100pt
\solution{
Since $\vec{M}$ is a constant, $\vec{J}_m = 0$. The red arrow indicate the direction of $\vec{J}_{ms}$ on each of the four sides. They each have a magnitude of $M_o$. The $\vec{J}_{ms}$ on the top and bottom surfaces are zero.\\
\begin{tikzpicture}
\node at (0,0) {\includegraphics[scale=0.5]{\wpath Q5-3.png}};
\draw [red,->>,line width=1pt] (-1,-0.6) -- (-1+0.9, -0.6+1.3);
\draw [red,->>,line width=1pt] (-1+0.9,-0.6+1.3) -- +(-2.3,0);
\draw [red,<<-,line width=1pt] (-3.4,-0.6) -- (-3.4+0.9, -0.6+1.3);
\draw [red,->>,line width=1pt] (-3.4, -0.7) -- +(2.3,0);
\end{tikzpicture}
}

\item By symmetry, determine what is the direction of the $\vec{B}$ field on a point on the $z$ axis.\\
\solution{
The $z$-component is non-zero. All other components are zero.
}

\item Determine the magnetic flux density vector at an arbitrary point on the $z$ axis.\\
\solution{
We add up the contributions of B-field from the four sides, and knowing that the current on each of the four sides is $M_o d$, to give
\begin{equation*}
\vec{B} = \frac{2 \sqrt{2} \mu_o M_o a^2 d}{\pi (4z^2 + a^2) \sqrt{2z^2 + a^2}} \vec{a}_z
\end{equation*}
}


\end{enumerate}


\answer{\begin{equation*}
\vec{B} = \frac{2 \sqrt{2} \mu_o M_o a^2 d}{\pi (4z^2 + a^2) \sqrt{2z^2 + a^2}} \vec{a}_z
\end{equation*}
}
\end{document}




































