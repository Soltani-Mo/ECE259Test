\def\showsolutions{true}
\def\showworkbooks{true}

\documentclass[../../header.tex]{subfiles}

\begin{document}
%Cheng 3-32. 
\textbf{Goal:} The radius of the core and the inner radius of the outer conductor of a very long coaxial transmission line are $r_i$ and $r_o$, respectively. The space between the conductors is filled with two coaxial layers of dielectrics. The dielectric constants of the dielectrics are $\varepsilon_{r1}$ for $r_i < r < b$ and $\varepsilon_{r2}$ for $b < r < r_o$. Determine its capacitance per unit length.\\
\\
\textbf{Steps:} 
\begin{enumerate}
\item Using Gauss's Law, determine an expression for the electric fields in both dielectrics.

\solution{
\begin{eqnarray*}
\vec{E}_1 &=& \vec{a}_{r} \frac{\rho_l}{2\pi \varepsilon_o\varepsilon_{r1} r}, \quad \text{for} \quad r_i < r < b\\
\vec{E}_2 &=& \vec{a}_{r} \frac{\rho_l}{2\pi \varepsilon_o\varepsilon_{r2} r}, \quad \text{for} \quad b < r < r_o\\
\end{eqnarray*}}

\item Determine an expression of the voltage between the two conductors.

\solution{
We integrate along the electric fields, yielding
\begin{equation*}
V = - \int_{r_o}^{r_i} \vec{E} \cdot d\vec{l} = \frac{\rho_l}{2\pi\varepsilon_o}
\left( \frac{1}{\varepsilon_{r1}} \ln \frac{b}{r_i} + \frac{1}{\varepsilon_{r2}} \ln \frac{r_o}{b} \right)
\end{equation*}}

\item Determine an expression for the capacitance per unit length.

\solution{
\begin{equation*}
C' = \frac{\rho_l}{V} = \frac{2\pi\varepsilon_o}{\frac{1}{\varepsilon_{r1}} \ln (b/r_i) + \frac{1}{\varepsilon_{r2}} \ln (r_o/b)}\;\text{(F/m)}
\end{equation*}}
\end{enumerate}
\answer{\begin{align*}
C^{'} = \frac{2\pi\varepsilon_o}{\frac{1}{\varepsilon_{r1}} \ln (b/r_i) + \frac{1}{\varepsilon_{r2}} \ln (r_o/b)}\;\text{(F/m)}
\end{align*}}
\end{document}




































