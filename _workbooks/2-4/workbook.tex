\def\showsolutions{true}
\def\showworkbooks{true}

\documentclass[../../header.tex]{subfiles}

\begin{document}
\textbf{Goal:} Find the work done by electric forces in moving a charge $Q = 1\nC$ from the coordinate origin to the point $(1\m,1\m,1\m)$ in the electrostatic field given by $\vect{E}(x,y,z) = (x \uvect{a}_x  + y^2 \uvect{a}_y - \uvect{a}_z)\, \text{V/m}$ along the straight line.\\
\\
\textbf{Steps:} 
\begin{enumerate}
\item Choose a path that would facilitate the computation of the work, based on the fact that 
$\int_C \vec{E} \cdot \d\vec{l}$ is path-independent. 

\solution{The parametric line describing the path from origin to point $(1\m,1\m,1\m)$ is given by:
\begin{equation*}
\vect{l}(t) = t \left( \uvect{a}_x + \uvect{a}_y + \uvect{a}_z\right)\;[\textrm{m}] \quad t \in [0,1]
\end{equation*}}{\vskip 80pt}

\item Compute scalar potential difference between point $(1\m,1\m,1\m)$ and the origin.

\solution{\begin{align*}
V &= -\int_{0}^{1} \vect{E}(x,y,z) \cdot d\vect{l}(t) \\
	&= - \int_{0}^{1} (x \uvect{a}_x  + y^2 \uvect{a}_y - \uvect{a}_z) \cdot \left( \uvect{a}_x + \uvect{a}_y + \uvect{a}_z \right) dt \\
 &=  0.1667\;[\textrm{V}]
\end{align*}
}{\vskip 80pt \newpage}

\item Use the potential to compute work done by electric forces, recalling the fundamental definition 
of electric potential difference as work per unit charge. 

\solution{\begin{align*}
W &= q V \\
& = 166.67\,\textrm{pJ}
\end{align*}
Therefore, the work done by the electric field is $-166.67\,\textrm{pJ}$.}{\vskip 80pt}
%\item Does the work done by electric forces change if we traverse a different path? Explain.

%\vskip 80pt
\end{enumerate}
\answer{\textit{W}=\,-\,166.67\,pJ}
\end{document}




































