\documentclass[../../header.tex]{subfiles}

\begin{document}
%\notaros{5-6}
\textbf{Goal:} \textit{Infinite cylinder with circular magnetization.} An infinitely-long ferromagnetic cylinder of radius $a$ in air has a nonuniform magnetization. In a cylindrical coordinate system whose $z$-axis coincides with the cylinder axis, $\vec{M} = M_o (r/a) \vec{a}_\phi$ ($0 \le r \le a$), where $M_o$ is a constant. Find the current densities $\vec{J}_m$ in the cylinder and $\vec{J}_{ms}$ on the cylinder, as well as the magnetic flux density $\vec{B}$ inside and outside the cylinder.\\
\\
\textbf{Steps:} 
\begin{enumerate}
\item Find the volume magnetization current density vector in the cylinder.\\
\solution{
Use the formula
\begin{equation*}
\vec{J}_m = \nabla \times \vec{M} = \frac{1}{r} \frac{\partial}{\partial r} \left( r M_\phi(z) \right) \vec{a}_z = \frac{2 M_o}{a} \vec{a}_z
\end{equation*}
}

\item Find the surface magnetization current density on the cylinder surface.\\
\solution{
Use the formula
\begin{equation*}
\vec{J}_{ms} = M_\phi(a^-)\vec{a}_\phi \times \vec{a}_r = -M_o \vec{a}_z.
\end{equation*}
}

\item Find the magnetic flux density vector in the cylinder.\\
\solution{
Because of symmetry, the $\vec{B}$-field in the cylinder (due to its magnetization currents assume to reside in a vaccum) is circular (magnetic-field lines are circles centered at the cylinder axis). To find the $\vec{B}$-field, we apply Amp\`{e}re's law as if the magnetization currents found in (a) and (b) were conduction currents in a nonmagnetic medium, to the circular contour $C$ of radius $r$, to give
\begin{equation*}
B 2 \pi r  = \mu_o J_m \pi r^2 \rightarrow \vec{B} = \frac{\mu_o J_m r}{2} \vec{a}\phi = \frac{\mu_o M_o r}{a} \vec{a}_\phi = \mu_o \vec{M}, \quad (0 \le r \le a)
\end{equation*}
}

\item Find the magnetic flux density vector outside the cylinder.\\
\solution{
For the observation point outside the cylinder, the right-hand side of Amp\`{e}re's law includes the surface magnetization current density $\vec{J}_{ms}$ as well, and this current amounts to $J_{ms}$ times the circumference of the cylinder. Hence, $\vec{B}$ outside the ferromagnetic cylinder is computed as
\begin{eqnarray*}
B 2 \pi r &=& \mu_o (J_m \pi a^2 + J_{ms} 2 \pi a)\\
B &=& \frac{\mu_o a}{r} \left( \frac{J_m a}{2} + J_{ms} \right)\\
&=& \frac{\mu_o a}{r} (M_o - M_o) = 0, \quad (a < r < \infty)
\end{eqnarray*}
}

\end{enumerate}
\answer{\begin{align*}
\vec{J}_m = \frac{2 M_o}{a} \vec{a}_z\\
\vec{J}_{ms} = -M_o \vec{a}_z\\
\vec{B} = \mu_o \vec{M}, \quad (0 \le r \le a)\\
\vec{B} = 0, \quad (a < r < \infty)
\end{align*}}
\end{document}




































