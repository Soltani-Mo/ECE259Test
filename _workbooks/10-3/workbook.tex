\documentclass[../../header.tex]{subfiles}

\begin{document}
%Notaros problem 6-16
\textbf{Goal:} \textit{Large square and small circular concentric coplanar loops.} Fig.~Q6.33 shows two concentric wire loops lying in the same plane, in free space. One is a large square loop of side length $a$ and the other is a small circular loop of radius $b$ ($b \ll a$). The loops are oriented in the same, counter-clockwise, direction. The square loop carries a low-frequency time-harmonic current of intensity $i(t) = I_o \sin \omega t$, and the resistance of the circular loop is $R$. Determine the induced current in the circular loop, neglecting its own magnetic field.
\begin{center}
\begin{tikzpicture}[scale=3]
\draw (-1,-1) rectangle(1, 1);
\draw [line width=1pt] (0,0) circle(0.2);
\draw [->, line width=1pt] (0,0) -- (0.2, 0);
\node [above] at (0.1, 0) {$b$};
\node [below] at (0,-1) {$a$};
\node [left] at (-1, 0) {$a$};
\draw [->, line width=1pt] (1,-0.1) -- (1, 0) node [left] {$i(t)$};
\end{tikzpicture}\\
\textbf{Figure 6.33} Magnetically-coupled large square and small circular loops in free space.
\end{center}
\textbf{Steps:} 
\begin{enumerate}
\item What is the $B$-field at the center of the square loop due to $i(t)$?
%\vskip 100pt

\solution{
The $B$-field points out of the page at the center of the square loop.
\begin{equation*}
B_\mathrm{center} = \frac{2\sqrt{2} \mu_o i}{\pi a}
\end{equation*}}

\item What is the magnetic flux through the circular loop?
%\vskip 80pt

\solution{
\begin{equation*}
\int \vec{B}\cdot d\vec{s} = B_\mathrm{center} S = B_\mathrm{center} = \frac{2\sqrt{2} \mu_o i}{a} b^2
\end{equation*}}

\item What is the emf induced?
%\vskip 200pt

\solution{
\begin{equation*}
e_\mathrm{ind}(t) = -\frac{d\Phi}{dt} = \frac{2\sqrt{2} \mu_o }{a} b^2 \frac{di}{dt}
\end{equation*}}

\item What is the induced current in the circular loop?
%\vskip 100pt

\solution{
\begin{equation*}
i_\mathrm{ind} = \frac{e_\mathrm{ind}}{R} =\frac{2\sqrt{2} \mu_o b^2 I_o}{a R} \frac{di}{dt} =\frac{2\sqrt{2} \mu_o b^2 \omega}{a R} \cos \omega t
\end{equation*}}

\end{enumerate}
\answer{\begin{equation*}
i_\mathrm{ind} = \frac{2\sqrt{2} \mu_o b^2 \omega}{a R} \cos \omega t
\end{equation*}}
\end{document}




































