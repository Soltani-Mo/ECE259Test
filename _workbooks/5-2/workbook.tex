\documentclass[../../header.tex]{subfiles}

\begin{document}
%Cheng 4-6.
\textbf{Goal:} Assume that the space between the inner and outer conductors of a long coaxial cylindrical structure is filled with an electron cloud having a volume density of charge $\rho = A / r$ for $\alpha < r < b$, where $\alpha$ and $b$ are the radius of the inner and outer conductor, respectively. The inner conductor is maintained at potential $V_0$ and the outer conductor is grounded.  Find $V(\alpha < r < b)$ by solving Poisson's equation.\\
\\
\textbf{Steps:} 
\begin{enumerate}
\item Choose coordinate system. \\
\solution{ Cylindrical }

\item Carefully state {\em Poisson} equation in the given space (note: there is charge density, in fact changing with $r$). \\
\solution{
\begin{align*}
\nabla^2 V &= -\frac{A}{\varepsilon r} \\
\frac{1}{r} \frac{\partial }{\partial r}\left( r \frac{\partial V}{ \partial r} \right) &= - \frac{A}{\varepsilon r}
\end{align*}}

\item State the boundary conditions for the potential.  \\
\solution{
\begin{align*}
V &= V_0 \quad \text{at $r = \alpha$} \\
V &= 0 \quad \text{at $r=b$}
\end{align*}}

\item Solve the differential equation, subject to boundary conditions. \\
%\vskip 150pt 
\solution{
\begin{align*}
V = -\frac{A r}{\varepsilon} + c_1 \ln r + c_2
\end{align*}
Applying the boundary conditions gives
\begin{align*}
V_0 &= -\frac{A}{\varepsilon} \alpha + c_1 \ln \alpha + c_2 \quad \text{at $r = \alpha $} \\
0 &= -\frac{A}{\varepsilon}b + c_1 \ln b + c_2 \quad \text{at $r = b$} \,.
\end{align*}
\begin{align*}
c_1 &= \frac{\frac{A}{\varepsilon} (b-\alpha) - V_0}{\ln (b/\alpha) } \\
c_2 &= \frac{V_0 \ln b + \frac{A}{\varepsilon} \left( \alpha \ln b - b \ln \alpha \right)}{\ln (b/\alpha)}
\end{align*}}

\item Verification step: Determine the electric field and verify that indeed it points in the direction of  decreasing potential. \\
\solution{
\begin{align*}
\vect{E} &= -\nabla V\\
&= -\frac{\partial V}{\partial r} \uvect{a}_r\\
&= \left [\frac{A}{\varepsilon} - \frac{c_1}{r} \right] \uvect{a}_{r} \\
&= \left[\frac{A}{\varepsilon} - {\frac{\frac{A}{\varepsilon} (b-\alpha) - V_0}{\ln (b/\alpha) r }} \right] \uvect{a}_r\,.
\end{align*}
The electric field $\vect{E}$ is equal to $-\frac{\partial V}{\partial r} \uvect{a}_r$. Since the outer conductor is grounded, the rate of change of potential with respect to the radial coordinate is negative; and the electric field points in $\uvect{a}_r$ direction. Hence, the electric field points in the direction of decreasing potential.}

\end{enumerate}
\answer{\begin{align*}
V = -\frac{A r}{\varepsilon} + c_1 \ln r + c_2\\
c_1 &= \frac{\frac{A}{\varepsilon} (b-\alpha) - V_0}{\ln (b/\alpha) } \\
c_2 &= \frac{V_0 \ln b + \frac{A}{\varepsilon} \left( \alpha \ln b - b \ln \alpha \right)}{\ln (b/\alpha)}
\end{align*}}
\end{document}



 
