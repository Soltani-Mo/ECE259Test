\documentclass[../../header.tex]{subfiles}

\begin{document}
%Notaros{6-13}
\textbf{Goal:} \textit{Moving contour near an infinite dc line current.} Assume that the current in the straight wire conductor from Fig.~Q6.12 is time-invariant, with intensity $I$, and that the contour moves away from the wire at a constant velocity $v$, as shown in Fig.~Q6.17. At $t = 0$, the distance of the closer parallel side of the contour from the wire is $x = c$. Determine the emf induced in the contour.
\begin{center}
\includegraphics[width=0.3\textwidth]{\wpath NotarosExample6-15.png}\\
\textbf{Figure 6.17} Evaluation of the emf in a rectangular contour moving in the magnetic field due to an infinitely long wire with a steady current.
\end{center}
\textbf{Steps:} 
\begin{enumerate}
\item What is the $B$-field everywhere due to the line current?
%\vskip 80pt

\solution{
\begin{equation*}
\vec{B}  = \frac{\mu_o I}{2 \pi r} \vec{a}_\phi
\end{equation*}}

\item What is the total magnetic flux through the contour?
%\vskip 200pt

\solution{
The position as a function of time is $x(t) = c + vt$. Then we integrate over the area of the contour.
\begin{equation*}
\Phi (t) = \frac{\mu_o I b}{2\pi} \ln \frac{x+a}{x} = \frac{\mu_o I b}{2\pi} \ln \frac{c + a + vt}{c + vt}
\end{equation*}}

\item What is the emf induced in the contour?
%\vskip 200pt

\solution{
\begin{equation*}
e_\mathrm{ind}(t) = -\frac{d\Phi}{dt} = -\frac{d\Phi}{dx} \frac{dx}{dt} = -\frac{d\Phi}{dx} v = \frac{\mu_o I ab v}{2\pi} \frac{1}{x(x + a)} = \frac{ \mu_o I ab v }{2\pi (c + vt)(c + a + vt)}
\end{equation*}}

\end{enumerate}
\answer{\begin{equation*}
e_\mathrm{ind}(t) = \frac{ \mu_o I ab v }{2\pi (c + vt)(c + a + vt)}
\end{equation*}}
\end{document}




































