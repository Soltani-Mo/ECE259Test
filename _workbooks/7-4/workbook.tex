\documentclass[../../header.tex]{subfiles}

\begin{document}
%Cheng 6-3: 
\textbf{Goal:} A current $I$ flows in the inner conductor of an infinitely long coaxial line and returns via the outer conductor. The radius of the inner conductor is $a$, and the inner and outer radius of the outer conductor are $b$ and $c$, respectively. Find the magnetic flux density $\vect{B}$ and plot is as a function of $r$ for $0 < r < c$.\\
\\
\textbf{Steps:} 
\begin{enumerate}
\item Determine an Amperian path for this structure.\\
%\vskip 20pt
\solution{
A circle of variable radius $r$.
}

\item What is the current density inside each conductor? Assume that the current is uniformly distributed inside each conductor.\\
%\vskip 20pt
\solution{
\begin{align*}
\vect{J}_1 &= \frac{I}{\pi a^2} \uvect{a}_z \,. \\
\vect{J}_2 &= \frac{-I}{\pi \left( c^2 - b^2\right) } \uvect{a}_z
\end{align*}
}

\item Apply Ampere's law to find the magnetic flux density $\vect{B}$ inside the inner conductor ($0<r<a$).\\
%\vskip 60pt
\solution{
\begin{align*}
\oint \vect{B} \cdot d\vect{l} &= \mu \int_0^{2\pi} \int_{0}^{r} \vect{J}_1 dS \\
2\pi r B_\phi  &= \mu \frac{I}{\pi a^2} \pi r^2 \\
\vect{B} &= \uvect{a}_\phi \frac{\mu r I}{2\pi a^2} 
\end{align*}
}

\item Apply Ampere's law to find $\vect{B}$ in the region between the two conductors ($a<r<b$).\\
%\vskip 80pt
\solution{
\begin{align*}
\oint \vect{B} \cdot d\vect{l} &= \mu I \\
\vect{B} &= \uvect{a}_\phi \frac{\mu I}{2\pi r} 
\end{align*}
}

\item Apply Ampere's law to find $\vect{B}$ in the outer conductor ($b<r<c$).\\
%\vskip 80pt
\solution{
\begin{align*}
\oint \vect{B} \cdot d\vect{l} &= \mu \left(I + \int_0^{2\pi} \int_b^r \vect{J}_2 r dr d\theta \right) \\
\vect{B} &= \uvect{a}_\phi \frac{\mu I}{2\pi r}\frac{c^2 - r^2}{c^2 - b^2}
\end{align*}
}

\item Use Ampere's law to compute $\vect{B}$ in the region outside the coaxial line ($r>c$).\\
%\vskip 40pt can be 
\solution{
\begin{align*}
\oint \vect{B} \cdot d\vect{l} &= 0\\
B_\phi  &= 0 \\
\end{align*}
There is no magnetic field outside the cable because there is zero net current enclosed by the Amperean loop.
}

\item Plot $\abs{\vect{B}}$ versus $r$ for all $0<r<c$. \\
\solution{
\includegraphics[scale = .7]{\wpath workbook4.png}
}

\end{enumerate}

\answer{\begin{enumerate}[label=(\alph*)]
\item \text{For} 0<r<a
\begin{align*}
\vect{B} &= \uvect{a}_\phi \frac{\mu r I}{2\pi a^2} 
\end{align*}
\item \text{For} b<r<c
\begin{align*}
\vect{B} &= \uvect{a}_\phi \frac{\mu I}{2\pi r}\frac{c^2 - r^2}{c^2 - b^2} 
\end{align*}
\item \text{For} r>c
\begin{align*}
B_\phi  &= 0
\end{align*}
\end{enumerate}
}

\end{document}




































