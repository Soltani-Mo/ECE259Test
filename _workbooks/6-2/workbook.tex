\documentclass[../../header.tex]{subfiles}

\begin{document}
%Cheng 6-4: 
\textbf{Goal:} A current $I$ flows lengthwise in a very long, thin conducting sheet of width $w$, as shown below. Assuming that the current flows into the paper, determine the magnetic flux density $\vect{B}$ at points $P_1(0,d)$ and $P_2(2w/3,d)$.
\begin{center}
\includegraphics[scale = 0.4]{\wpath workbook4.png}
\end{center}
\textbf{Steps:} 
\begin{enumerate}

\item The problem can be solved with superposition. What is the differential current element? \\
\solution{
\begin{align*}
d\vect{I}' =  \frac{I}{w} d\vect{l}' = - \frac{I}{w} dx' dz' \uvect{a}_z
\end{align*}
}

\item Determine the observation vector $\vect{R}$\\
\solution{
\begin{align*}
\vect{R} = d \vect{a}_y
\end{align*}
}

\item Determine the source vector $\vect{R}'$\\
\solution{
\begin{align*}
\vect{R}' = x' \uvect{a}_x + z' \uvect{a}_z
\end{align*}
}

\item What is the differential magnetic flux density $d\vect{B}$\\
\solution{
\begin{align*}
d\vect{B} &= \frac{\mu_0 I}{4\pi w}  d\vect{l}' \times (\vect{R} - \vect{R}') \\
&= \frac{\mu_0 I}{4\pi w}  \frac{d \, dz' dx' \uvect{a}_x + x' dz' dx' \uvect{a}_y}{(d^2 + x'^2 + z'^2)^{3/2}}\\
\end{align*}
}

\item Solve the superposition integral.\\
\solution{
\begin{align*}
\vect{B} &= \frac{\mu_0 I}{4\pi w} \int_0^w \int_{-\infty}^{\infty} \frac{d \, dz' {d}x' \uvect{a}_x + x' dz' dx' \uvect{a}_y}{(d^2 + x'^2 + z'^2)^{3/2}} \\
&=\frac{\mu_0 I}{4\pi w}  \int_0^w \left( d  \uvect{a}_x + x'  \uvect{a}_y \right)\left(\int_{-\infty}^{\infty} \frac{dz}{(d^2 + x'^2 + z'^2)^{3/2}} \right)dx' \\
&= \frac{\mu_0 I}{2\pi w}  \int_0^w \left( d  \uvect{a}_x + x'  \uvect{a}_y \right) \left( \frac{1}{d^2 + x'^2}\right) dx' \\
&= \frac{\mu_0 I}{2\pi w} \tan^{-1} \left(\frac{w}{d} \right) \uvect{a}_x + \frac{\mu_0 I}{4\pi w} \ln \left(1 + \frac{w^2}{d^2} \right) \uvect{a}_y
\end{align*}
}

\item Use the answer of part (5) to find magnetic flux density $\vect{B}_2$ at point $P_2(2w/3, d)$.\\
\solution{
Magnetic flux density due to current strip to the right (R) and to left are given by (L)
\begin{align*}
&\vect{B}_{L} = \frac{\mu_0 I}{2\pi w} \tan^{-1} \left(\frac{2w}{3d} \right) \uvect{a}_x - \frac{\mu_0 I}{4\pi w} \ln \left(1 + \frac{4w^2}{9d^2} \right) \uvect{a}_y \\
&\vect{B}_{R} = \frac{\mu_0 I}{2\pi w} \tan^{-1} \left(\frac{w}{3d} \right) \uvect{a}_x + \frac{\mu_0 I}{4\pi w} \ln \left(1 + \frac{w^2}{9d^2} \right) \uvect{a}_y \\
&\vect{B} = \vect{B}_L + \vect{B}_R
\end{align*}
}
\end{enumerate}

\answer{
\begin{align*}
\vect{B_1} &=\frac{\mu_0 I}{2\pi w} \tan^{-1} \left(\frac{w}{d} \right) \uvect{a}_x + \frac{\mu_0 I}{4\pi w} \ln \left(1 + \frac{w^2}{d^2} \right) \uvect{a}_y\\
&\vect{B}_{L} = \frac{\mu_0 I}{2\pi w} \tan^{-1} \left(\frac{2w}{3d} \right) \uvect{a}_x - \frac{\mu_0 I}{4\pi w} \ln \left(1 + \frac{4w^2}{9d^2} \right) \uvect{a}_y \\
&\vect{B}_{R} = \frac{\mu_0 I}{2\pi w} \tan^{-1} \left(\frac{w}{3d} \right) \uvect{a}_x + \frac{\mu_0 I}{4\pi w} \ln \left(1 + \frac{w^2}{9d^2} \right) \uvect{a}_y \\
&\vect{B_2} = \vect{B}_L + \vect{B}_R
\end{align*}}

\end{document}

