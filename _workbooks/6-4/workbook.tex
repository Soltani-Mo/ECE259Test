\documentclass[../../header.tex]{subfiles}

\begin{document}
%Cheng 6-11.
\textbf{Goal:} A long wire carrying a current $I$ folds back with a semicircular bend of radius $b$ as in Fig.~Q6-38. Determine the magnetic flux density $\vec{B}$ at the center point $P$ of the bend.
\begin{center}
\begin{tikzpicture}
\draw (0,1) -- (5,1);
\draw (0,-1) -- (5,-1);
\begin{scope}
\clip (-1,-1) rectangle (0,1);
\draw (0,0) circle(1);
\end{scope}
\node [right] at (0,0) {$P$};
\fill (0,0) circle (0.05);
\draw [->,thick] (2,1) -- (1.99,1);
\draw [->,thick] (2,-1) -- (2.01,-1);
\draw [->] (0,0) -- (-1,0) node[pos=0.5,above] {$b$};
\node [above] at (2,1) {$I$};
\node [below] at (2,-1) {$I$};
\end{tikzpicture}\\
Fig.~Q6-38
\end{center}
\textbf{Steps:} 
\begin{enumerate}
\item Determine the contribution of $B$ due to the two straight wires. Hint: exploit symmetry\\
\solution{
\begin{equation*}
\vec{B} = \frac{\mu_o I}{2\pi b} \vec{a}_z
\end{equation*}
}

\item Determine the contribution of $\vec{B}$ due to the semi-circle.\\
\solution{
\begin{equation*}
\vec{B} = \frac{\mu_o I}{4 b} \vec{a}_z
\end{equation*}
}

\item Determine the total $\vec{B}$ field at point $P$.\\
\solution{
\begin{equation*}
\vec{B} = \frac{\mu_o I} {2 b} \left( \frac{1}{\pi} + \frac{1}{2} \right) \vec{a}_z
\end{equation*}
}

\end{enumerate}
\answer{\begin{equation*}
\vec{B} = \frac{\mu_o I} {2 b} \left( \frac{1}{\pi} + \frac{1}{2} \right) \vec{a}_z
\end{equation*}}
\end{document}






