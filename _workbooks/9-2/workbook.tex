\documentclass[../../header.tex]{subfiles}

\begin{document}
%Cheng 7-3
\textbf{Goal:} A rectangular loop of width $w$ and height $h$ is situated near a very long wire carrying a current $i_1(t)$, as shown in the figure below. Assume $i_1$ to be a step function as shown in the right panel of the figure below. Find the current $i_2(t)$ induced in the rectangular loop as a function of the mutual inductance $L_{12}$ between the two circuits, the self-inductance $L_{22}$ of the square loop, and the resistance $R$ (note: you don't have to calculate $L_{12}$ and $L{_22}$).
\begin{center}
%\includegraphics[width=0.3\textwidth]{./Cheng7-3a.png}
%\includegraphics[width=0.4\textwidth]{./Cheng7-3b.png}\\
%\textbf{Figure 7-11}. A rectangular loop near a long current-carrying wire.
\begin{tikzpicture}[scale=1.5]

	\draw [->] (1,0) -- (1,2) node [left] {$i_1(t)$};
	\draw (1,2) -- (1,4);
	
	\draw [->] (2,2) node [left] {$i_2(t)$} -- (2,3) -- (2.6,3) -- (2.7,3.2) -- (2.9,2.8) -- (3.1,3.2) node [above] {$R$} -- (3.3,2.8) -- (3.5,3.2) -- (3.6,3) -- (4,3) -- (4,1) -- (2,1) -- (2,2);

	\draw [<->,dashed] (1,0.5) -- (2,0.5) node[midway,below] {$d$};
	\draw [<->,dashed] (2,0.5) -- (4,0.5) node[midway,below] {$w$};
	\draw [<->,dashed] (4.5,1) -- (4.5,3) node[midway,right] {$h$};		
	
	
	\draw [->] (6,1) -- (9,1) node [below] {$t$};
	\draw [->] (7,0.5) -- (7,2.5) node [left] {$i_1(t)$};	
	\draw [very thick] (6,1) -- (7,1) -- (7,2) node [left] {$I_1$} -- (9,2);
\end{tikzpicture}

\end{center}
\textbf{Steps:} 
\begin{enumerate}
\item Write the magnetic flux $\Phi_2$ through the rectangular loop caused by currents $i_1(t)$ and $i_2(t)$. Express the flux as a function of $L_{12}$ and $L_{22}$.

\solution{
\begin{eqnarray*}
\Phi_2 &=& L_{12} i_1(t) + L_{22} i_2(t)
\end{eqnarray*}}

\item Use Faraday's law to show that $i_2(t)$ satisfies the differential equation
\begin{equation*}
R i_2(t) + L_{22} \frac{d i_2}{dt} = - L_{12} \frac{d i_1}{dt}
\end{equation*}
\\

\solution{
Defining the points ABCD as shown:
\begin{center}
\includegraphics[width=0.3\textwidth]{\wpath workbook2_solution_1.png}
\end{center}
Faraday's law on $C_2$
\begin{eqnarray*}
\oint_{C_2} \vec{E}\cdot d\vec{l} &=& -\frac{d}{dt} \Phi_2(t) = - L_{12}\frac{di_1}{dt} - L_{22} \frac{di_2}{dt}\\
&=&\int_{BCDA} \vec{E}\cdot d\vec{l} + \int_{A \rightarrow B} \vec{E} \cdot d\vec{l} = 0 - \underbrace{\int_B^A \vec{E} \cdot d\vec{l}}_{\text{Voltage across R}} = R i_2(t)
\end{eqnarray*}
Hence,
\begin{equation*}
R i_2(t) = - L_{12} \frac{d i_1}{dt} - L_{22} \frac{d i_2}{dt}
\end{equation*}}

\item Solve the differential equation to find $i_2(t)$ (you can use the Laplace transform method).

\solution{
The $i_1(t)$ is given by
\begin{equation*}
i_1(t) = I_1 U(t)
\end{equation*}
where $U(t)$ is the unit-step function. Taking the Laplace transform of
\begin{equation*}
R i_2(t)  + L_{22} \frac{d i_2}{dt} = - L_{12} \frac{d i_1}{dt}
\end{equation*}
to get
\begin{eqnarray*}
RI_2(s) &+& s L_{22} I_2(s) = -L_{12} I_1\\
I_2(s) &=& \frac{ L_{12} I_1 }{s L_{22} + R} = -\frac{L_{12} I_1}{L_{22}} \frac{1}{s + \frac{R}{L_{22}} }
\end{eqnarray*}
Taking the inverse Laplace transform of $I_2(s)$ to get
\begin{equation*}
i_2(t) = -\frac{L_{12} I_1}{L_{22}}\,\e^{-\frac{R}{L_{22}} t} \quad \text{for} \quad t \ge 0
\end{equation*}}

\end{enumerate}
\answer{
\begin{equation*}
i_2(t) = -\frac{L_{12} I_1}{L_{22}}\,\e^{-\frac{R}{L_{22}} t} \quad \text{for} \quad t \ge 0
\end{equation*}
}
\end{document}




































