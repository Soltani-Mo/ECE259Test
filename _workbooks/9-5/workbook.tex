\documentclass[../../header.tex]{subfiles}

\begin{document}
%Cheng 6-46
\textbf{Goal:} The bar AA' in the figure below serves as a conducting path (such as the blade of a circuit breaker) for the current $I$ in two very long (semi-infinite) parallel lines. The lines have a radius $b$ and are spaced at a distance $d$ apart. Find the direction and the magnitude of the magnetic force on the bar.
\begin{center}
\includegraphics[scale = 0.6]{\wpath Q6-46.png}
\end{center}
\textbf{Steps:} 
\begin{enumerate}
\item In order to compute the force, first determine the magnetic flux density $\vect{B}_{1}$  at point $(0,y)$ due to the current in line \#1. Use Biot-Savart law to find $\vect{B}_1$.
\begin{enumerate}
\item What is $\vect{R} - \vect{R}'$?
%\vskip 20pt

\solution{
\begin{align*}
\vect{R} - \vect{R}' = y \uvect{a}_y - x' \uvect{a}_x
\end{align*}}

\item What is the differential current $d\vect{I}'$?
%\vskip 20pt

\solution{
\begin{align*}
d\vect{I}' = -I dx' \uvect{a}_x
\end{align*}}

\item What is differential magnetic flux density $d\vect{B}_1$?
%\vskip 20pt

\solution{
\begin{align*}
d\vect{B}_1 &= \frac{\mu_0 I}{4\pi} \left(\frac{-\uvect{a}_x dx' \times \left(y\uvect{a}_y - x' \uvect{a}_x  \right) }{ (y^2 + (x')^2 )^{3/2} } \right) \\
&= \frac{\mu_0 I}{4\pi} \frac{-y dx'\uvect{a}_z }{(y^2 + (x')^2)^{3/2}}
\end{align*}}

\item What is $\vect{B}_1$?
%\vskip 20pt

\solution{
\begin{align*}
\vect{B}_1 &= -\uvect{a}_z \frac{\mu_0 I}{4\pi} \int_0^{\infty} \frac{y}{(y^2 + (x')^2)^{3/2}}dx'\\
&=-\uvect{a}_z \frac{\mu_0 I}{4\pi y}
\end{align*}}
\end{enumerate}

\item Use results in part (1) to find the magnetic flux density $\vect{B}_2$ at point $(0,y)$ due to line \#2.
%\vskip 30pt
\begin{align*}
\vect{B}_2 = -\uvect{a}_z \frac{\mu_0 I}{4\pi (d-y)}
\end{align*}


\item What is the  force  $d\vect{F}$ acting on a short section $dy$ of the bar AA'? Let $(0,y)$ be the position of the differential section.
%\vskip 30pt

\solution{
\begin{align*}
d\vect{F} &= I dy \uvect{a}_y  \times \left(\vect{B}_1 + \vect{B}_2 \right) \\
&= \frac{\mu_0 I^2 dy}{4\pi} \left( \uvect{a}_y \times \uvect{a}_z \left( -\frac{1}{y}  - \frac{1}{d-y} \right)\right) \\
&= -\uvect{a}_x \frac{\mu_0 I^2 }{4\pi} \left(\frac{1}{y} + \frac{1}{d-y} \right)dy
\end{align*}}

\item Integrate $d\vect{F}$ to find the total force on the bar.
%\vskip 40pt

\solution{
\begin{align*}
\vect{F} &= -\uvect{a}_x \frac{\mu_0 I^2 }{4\pi} \int_{b}^{d-b} \left(\frac{1}{y} + \frac{1}{d-y} \right)dy \\
&= -\uvect{a}_x \frac{\mu_0 I^2 }{4\pi} \left[ \ln \left(\frac{d-b}{b} \right) - \ln \left( \frac{b}{d-b} \right) \right] \\
&= -\uvect{a}_x \frac{\mu_0 I^2 }{2\pi} \ln \left(\frac{d}{b} - 1 \right)
\end{align*}}
\end{enumerate}
\answer{\begin{align*}
\vect{F}&= -\uvect{a}_x \frac{\mu_0 I^2 }{2\pi} \ln \left(\frac{d}{b} - 1 \right)
\end{align*}}
\end{document}









