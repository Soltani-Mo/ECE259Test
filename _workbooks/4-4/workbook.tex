\def\showsolutions{true}
\def\showworkbooks{true}

\documentclass[../../header.tex]{subfiles}

\begin{document}
\textbf{Goal:} A capacitor consists of two coaxial metallic cylindrical surfaces of a length $L = 30\mm$ and radii $r_1 = 5\mm$ and $r_2 = 7\mm$. The dielectric material between the surfaces has a relative permittivity $\varepsilon_{r} = 2 + (4/r)$, where $r$ is measured in mm. Determine the capacitance.\\
\\
\textbf{Steps:} 

\begin{enumerate}
\item Use Gauss' law to compute the electric field inside the dielectric due to charge $+Q$ on the inner conductor and $-Q$ on the outer conductor.

\solution{
\begin{align*}
\vect{E} &= \uvect{a}_r \frac{Q}{2\pi \varepsilon r L} \\
&= \vect{a}_r \frac{Q}{4\pi \varepsilon_0 (r + 2) L}
\end{align*}}

\item Compute potential difference between the two conductors.

\solution{
\begin{align*}
V &= - \int_{r_o}^{r_i} \vect{E} \cdot d\vect{r} \\
&= \frac{Q}{4\pi \varepsilon_0 L} \ln (r+2) \biggl |^{7}_{5} \\
&= \frac{Q}{4 \pi \varepsilon_0 L} \ln \left (\frac{9}{7}\right) \,.
\end{align*}}

\item What is the capacitance of this cylindrical capacitor?

\solution{
\begin{align*}
C &= \frac{Q}{V} \\
&= \frac{4 \pi \varepsilon_0 L}{\ln \left( \frac{9}{7} \right)} \\
&= 1500 \varepsilon_0 \\
&= 13.26\,\text{\textmu{}F}\,.
\end{align*}}
\end{enumerate}
\answer{C=13.26\,$\mu$F}
\end{document}




































