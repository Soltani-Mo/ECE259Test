\documentclass[../../header.tex]{subfiles}

\begin{document}
%Notaros example 6-9
\textbf{Goal:} \textit{Rectangular contour near an infinite line current.} An infinitely long straight wire carries a slowly time-varying current density of intensity $i(t)$. A rectangular contour of side lengths $a$ and $b$ lies in the same plane with the wire, with two sides parallel to it, as shown in Fig.~6.12. The distance between the wire and the closer parallel side of the contour is $c$. Determine the emf induced in the contour.
\begin{center}
\includegraphics[width=0.3\textwidth]{\wpath NotarosExample6-9.png}\\
\textbf{Figure 6.12} Evaluation of the emf in a rectangular contour in the vicinity of an infinitely long wire with a slowly time-varying current.
\end{center}
\textbf{Steps:} 
\begin{enumerate}
\item State the integral form of Faraday's law.
%\vskip 80pt

\solution{
\begin{equation*}
\oint_C \vec{E} \cdot d\vec{l} = -\frac{\partial \Phi}{\partial t},
\end{equation*}
where $\Phi$ is the total magnetic flux through a surface with contour $C$.}

\item What is the $\vec{B}$-field everywhere due to $i(t)$?
%\vskip 100pt

\solution{
The $B$-field is into the page.
\begin{equation*}
B(x, t) = \frac{\mu_o i(t)}{2 \pi x}
\end{equation*}}

\item What is $\Phi(t)$, the magnetic flux through the loop?
%\vskip 200pt

\solution{
We integrate over the area of the loop.
\begin{equation*}
\Phi(t) = \int_c^{c+a} B(x,t) b\,dx = \frac{\mu_o i(t) b}{2\pi} \int_c^{c+a} \frac{dx}{x} = \frac{\mu_o i(t) b}{2 \pi} \ln \frac{ c+a }{c}
\end{equation*}}

\item What is the emf induced in the loop?
%\vskip 100pt

\solution{
\begin{equation*}
e_{ind}(t) = -\frac{d \Phi}{dt} = -\frac{\mu_o b}{2\pi} \ln \frac{c+a}{c} \frac{di}{dt}
\end{equation*}}

\end{enumerate}
\answer{\begin{equation*}
e_{ind}(t) = -\frac{\mu_o b}{2\pi} \ln \frac{c+a}{c} \frac{di}{dt}
\end{equation*}}
\end{document}




































