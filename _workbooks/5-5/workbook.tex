\documentclass[../../header.tex]{subfiles}

\begin{document}
%Cheng 5-17. 
\textbf{Goal:} Find the resistance between the surfaces $R_1$ and $R_2$ of a truncated conical block defined by $R_1 \leq R \leq R_2$ and $0 \leq \theta \leq \theta_0$. The two spherical surfaces ($R=R_1$ and $R=R_2$) are perfect electric conductors (PECs), while the rest of the block has conductivity $\sigma$. You can neglect edge effects.\\
\\
\textbf{Steps:} 
\begin{enumerate}
\item Choose a coordinate system. \\
\solution{Spherical}

\item Assume potential $V_0$ on the one conductor and zero on the other. From  Laplace equation, find $V$ and then $E$. 

\textit{Note: Alternatively, you can assume a charge $Q$ on the inner conductor and use Gauss' law to find the field it creates}\\
\solution{
Solving Laplace's equation assuming scalar potential of $V_0$ on inner surface and $0\ V$ on outer surface we get
\begin{align*}
\nabla^2 V &= 0 \\
\frac{1}{R^2} \frac{\partial }{\partial R} \left(R^2 \frac{\partial V}{\partial R}\right) &= 0\\
\end{align*}
The solution to this equation is
\begin{align*}
V = -\frac{c_1}{R} + c_2
\end{align*}
Applying the boundary conditions ($V(R_1) = V_0$ and $V(R_2) = 0$) gives:
\begin{align*}
c_1 &= \frac{V_0}{\frac{1}{R_2} - \frac{1}{R_1} } \\
c_2 &= \frac{1}{R_2} \frac{V_0}{\frac{1}{R_2} - \frac{1}{R_1} }
\end{align*}
}
\solution{
From scalar potential,
\begin{align}
\vect{E} &= - \nabla V \nonumber \\ 
&= \left( \frac{V_0}{\frac{1}{R_1} - \frac{1}{R_2}} \right)\frac{1}{R^2} \uvect{a}_R\,. 
\end{align}\\
\\
\textit{Alternatively, on a Gaussian surface of a cone with $0\le \theta \le \theta_0$ and radius $R$ we apply Gauss' law:
\begin{align}
	\int_0^{2\pi} \int_{0}^{\theta_0} \vect{E} \cdot \uvect{a}_R R^2 \sin\theta\,d\theta\,d\phi &= \frac{Q}{\varepsilon} \nonumber  \\
	2 \pi R^2 E_R (1-\cos \theta)  &= \frac{Q}{\varepsilon} \nonumber \\
	E_R &= \frac{Q}{2\pi \varepsilon R^2 (1- \cos \theta_0)} \,. 
\end{align}
The potential is given by:
\begin{align}
	V_0 &= -\int_{R_2}^{R_1} \frac{Q}{2\pi \varepsilon R^2 (1 - \cos \theta_0)} dR \nonumber \\
	&= \frac{Q}{2\pi \varepsilon (1-\cos\theta_0)} \left(\frac{1}{R_1} - \frac{1}{R_2}\right)
\end{align}
Note: the electric field obtained from the Laplace equation and Gauss' law are equivalent.}}

\item Having $\vect{E}$, find $\vect{J}$. Can you find the total current $I$ that this $\vect{J}$ creates? Choose the surface that you need to use to apply the formula: 
$I = \iint_S \vect{J} \cdot \vect{dS}$. \\
\solution{
\begin{align*}
\vect{J} &= \sigma \vect{E} \\
J_R &= \sigma \left( \frac{V_0}{\frac{1}{R_1} - \frac{1}{R_2}} \right)\frac{1}{R^2}
\end{align*}
Using conic surface, we can compute the total current I:
\begin{align*}
I &= \int_0^{2\pi} \int_0^{\theta_0} J_R R^2 \sin \theta\,d\theta\,d\phi \\
&= \sigma \left( \frac{V_0}{\frac{1}{R_1} - \frac{1}{R_2}} \right) 2 \pi (1 - \cos \theta_0)
\end{align*}
}

\item Having $\vect{E}$, find the voltage between the conductors. Then, $R=V/I$. 
\solution{
\begin{align*}
R &= \frac{V_0}{I} \\
&= \frac{\frac{1}{R_1} - \frac{1}{R_2}}{ 2 \pi \sigma  (1 - \cos \theta_0)}
\end{align*}
}

\end{enumerate}
\answer{\begin{align*}
R = \frac{\frac{1}{R_1} - \frac{1}{R_2}}{ 2 \pi \sigma  (1 - \cos \theta_0)}
\end{align*}}
\end{document}




































