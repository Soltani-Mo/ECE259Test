\def\showsolutions{true}
\def\showworkbooks{true}

\documentclass[../../header.tex]{subfiles}

\begin{document}
%Cheng 3-33.
\textbf{Goal:} A cylindrical capacitor of length $L$ consist of coaxial conducting surfaces of radii $r_i$ and $r_o$. Two dielectric media of different dielectric constants $\varepsilon_{r1}$ and $\varepsilon_{r2}$ fill the space between the conducting surfaces as shown in Fig.~3-42. Determine its capacitance.\\
\\
\textbf{Steps:} 
\begin{enumerate}
\item Is the electric fields in both of the dielectric same or different? Think about the boundary conditions.

\solution{
The boundary condition tangential E-fields are continuous across dielectric-dielectric boundaries. The E-fields are the same on both dielectric regions.}

\item Using Gauss's Law, obtain an expression for the electric field.

\solution{
\begin{equation*}
\pi r L \varepsilon_o \varepsilon_{r1} E_r + \pi r L \varepsilon_o \varepsilon_{r2} E_r = \rho_l L
\end{equation*}
Isolating for $E_r$ to obtain
\begin{equation*}
E_r = \frac{\rho_l}{\pi r \varepsilon_o (\varepsilon_{r1} + \varepsilon_{r2}) }
\end{equation*}}

\item Obtain an expression for the voltage between the two conductors.

\solution{
\begin{equation*}
V = -\int_{r_o}^{r_i} E_r\,dr = \frac{\rho_l}{\pi \varepsilon_o (\varepsilon_{r1} + \varepsilon_{r2})} \ln \frac{r_o}{r_i}
\end{equation*}}

\item Obtain an expression for the capacitance.

\solution{
\begin{equation*}
C = \frac{\rho_l L}{V} = \frac{\pi \varepsilon_o(\varepsilon_{r1} + \varepsilon_{r2}) L}{\ln (r_o/r_i)}
\end{equation*}}

\end{enumerate}
\answer{\begin{align*}
C = \frac{\pi \varepsilon_o(\varepsilon_{r1} + \varepsilon_{r2}) L}{\ln (r_o/r_i)}
\end{align*}}
\end{document}




































